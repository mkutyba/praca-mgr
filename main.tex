\documentclass[twoside]{iisthesis}
\usepackage[MeX]{polski}
\usepackage[cp1250]{inputenc}
\usepackage{graphicx}

% definicje kolorow
\definecolor{ciemnoSzary}{rgb}{0.15,0.15,0.15}
\definecolor{szary}{rgb}{0.5,0.5,0.5}
\definecolor{jasnoSzary}{rgb}{0.2,0.2,0.2}

\begin{document}

% Zmiana domy�lnych angielskich nazw cz�ci dokumentu na polskie "Rozdzia�y s� w klasie issthesis"
% wst�pnie poprawione mo�e kto� b�dzie zna� lepszy spos�b i wrzuci to do klasy issthesis.cls
\renewcommand{\contentsname}{Spis tre�ci}
\renewcommand{\appendixname}{Dodatek}
\renewcommand{\listfigurename}{Spis ilustracji}
\renewcommand{\listtablename}{Spis tabel}
\renewcommand{\refname}{Bibliografia}
\renewcommand{\abstractname}{Streszczenie}

\title{Predykcja defekt�w na poziomie metod w celu zredukowania wysi�ku zwi�zanego z zapewnieniem jako�ci oprogramowania}
\author{Mateusz Kutyba}
\advisor{dr hab. in�. Lech Madeyski}
\instituteLogo{logos/pwr}
\slowaKluczowe{pierwsze\\drugie\\trzecie}

\date{\number\the\year}

% Wstawienie abstractu pracy
\abstractSH{Bardzo kr�tkie streszczenie w kt�rym powinno si� znale�� om�wienie tematu pracy i poruszanych termin�w. Tekst ten nie mo�e by� zbyt d�ugi.}
\abstractPL{Streszczenie po polsku}
\abstractEN{Abctract in english}

%spis tre�ci
\maketitle
\textpages

%%%%%%%%%%%%%%%%%%%%%%%%%%%%%%%%%%%%%%%%%%%%%%%%%%%%%%%%%%%%
\chapter{Wst�p}
\noindent
		\noindent
T�o: r�ne podej�cia do predykcji defekt�w, potrzeba gromadzenia danych z projekt�w.

Motywacja: istniej�ce modele skupiaj� si� na predykcji na poziomie klas, pakiet�w, plik�w. Przegl�d kodu du�ych projekt�w jest kosztowny.

Dodatkowo gromadzenie danych z projekt�w jest czasoch�onne, wymaga du�o pracy i pobierania projekt�w, potrzeba stworzenia uniwersalnych rozwi�za� s�u��cych do tego celu oraz nastawienie na mo�liwo�� rozszerzania zestawu narz�dzi, kt�re mog� by� ze sob� dowolnie zestawiane




	\section{Cel pracy}
	\noindent
Stworzenie wtyczek do Knime, s�u��cych do wyliczania metryk procesu na podstawie danych z system�w kontroli wersji, zgromadzenie metryk z projekt�w Open Source w repozytorium.

Za�o�enia: wtyczki w �rodowisku Knime, wykorzystanie istniej�cych rozwi�za� platformy DePress, systemy kontroli wersji: git, svn.

Ograniczenia zakresu tematycznego: badanie tylko projekt�w pisanych w Java

	\section{Zwi�zek z innymi pracami}
	\noindent
\begin{itemize}
	\item przegl�d literatury
	\item uzasadnienie, �e nie tworz� czego�, co wcze�niej stworzyli inni
	\item tw�rcze wykorzystanie do�wiadcze� (wynik�w pracy) innych
\end{itemize}





Wst�p Lorem \cite{bramwell2001} ipsum dolor sit amet, consectetur adipiscing elit. Integer odio lectus, egestas sed aliquet ut, semper et tortor. Ut consectetur augue non\cite{padfield2007}metus scelerisque accumsan. Ut quis odio dolor. Suspendisse eu ante ut lorem sagittis lacinia. Mauris elit neque, eleifend eget varius ut, adipiscing sit amet quam. Duis sit amet ullamcorper purus. Aliquam erat volutpat. Donec pharetra massa sed mauris vehicula aliquet eget eu nulla. Etiam non felis metus. Etiam consequat commodo ante. Cras accumsan risus eu enim volutpat quis lobortis risus sagittis. Mauris posuere rutrum ante, at consectetur massa iaculis in.

Duis tristique, sapien et posuere iaculis, sapien lectus ultricies risus, nec placerat risus diam varius arcu. Curabitur eget metus id metus placerat dignissim. Curabitur faucibus est erat, at fermentum lorem. Sed interdum imperdiet erat, vitae dapibus eros feugiat non. Donec sit amet massa nec tortor elementum blandit. Vivamus eget magna odio. Donec hendrerit diam scelerisque arcu porta ut tempus lorem vulputate. Mauris sed tellus in risus bibendum luctus eu et sapien. In et scelerisque purus. Ut vel enim dolor. Proin vitae ipsum odio, nec pretium enim.

Nunc id ligula at felis vestibulum accumsan. Sed molestie erat eu nibh semper tincidunt. In iaculis nisl sed sem dictum ut viverra velit tincidunt. Nunc eros elit, dapibus adipiscing iaculis ut, facilisis sit amet arcu. In blandit, libero vel ornare suscipit, orci eros egestas nisi, ac tincidunt arcu diam non nulla. Mauris auctor, diam ac cursus scelerisque, diam lorem vulputate urna, ut fringilla turpis nunc eget felis. In hac habitasse platea dictumst. Nunc id leo libero. Sed nunc nunc, venenatis sit amet semper mollis, lacinia non purus. Vestibulum id fermentum diam. Suspendisse faucibus massa ac sapien fermentum quis commodo ipsum tempus. Duis sagittis consectetur mi a tincidunt. In quam eros, viverra quis malesuada a, iaculis ac neque. Etiam pharetra, odio varius pharetra interdum, mauris dolor sodales leo, a sagittis enim orci in nulla. Nunc eu quam ligula, nec iaculis tortor.

Sed suscipit interdum vulputate. Nulla vitae accumsan eros. Etiam eu consectetur enim. Duis et erat ut neque vestibulum malesuada. Cras et lectus dolor. Nullam feugiat semper mattis. Suspendisse non justo aliquet risus mattis tristique nec ut orci. Nulla facilisi. Ut bibendum neque sed nibh egestas a cursus mi volutpat. Morbi ac mi a enim tristique cursus. Pellentesque vitae diam massa. Donec dignissim blandit sollicitudin. Nulla dignissim dolor est, ut dictum libero. Ut sagittis tempus semper. Morbi vitae convallis dui. Quisque ut quam massa.

Nunc bibendum orci id nisl posuere sit amet egestas nisl dictum. Cras faucibus hendrerit elit, non suscipit lacus pharetra in. Vivamus eu libero quis risus laoreet egestas. Morbi eleifend aliquet nisi id semper. Cras consequat est id urna malesuada scelerisque. Aliquam tristique, orci vitae suscipit imperdiet, enim magna laoreet leo, in dignissim mi elit vitae lacus. Morbi ut enim vel enim pretium elementum. Maecenas ultricies blandit nisl, eget tincidunt ligula cursus mattis. Fusce et massa non risus euismod gravida at quis quam. Duis sed nibh justo, id semper justo.

Duis tristique, sapien et posuere iaculis, sapien lectus ultricies risus, nec placerat risus diam varius arcu. Curabitur eget metus id metus placerat dignissim. Curabitur faucibus est erat, at fermentum lorem. Sed interdum imperdiet erat, vitae dapibus eros feugiat non. Donec sit amet massa nec tortor elementum blandit. Vivamus eget magna odio. Donec hendrerit diam scelerisque arcu porta ut tempus lorem vulputate. Mauris sed tellus in risus bibendum luctus eu et sapien. In et scelerisque purus. Ut vel enim dolor. Proin vitae ipsum odio, nec pretium enim.

Nunc id ligula at felis vestibulum accumsan. Sed molestie erat eu nibh semper tincidunt. In iaculis nisl sed sem dictum ut viverra velit tincidunt. Nunc eros elit, dapibus adipiscing iaculis ut, facilisis sit amet arcu. In blandit, libero vel ornare suscipit, orci eros egestas nisi, ac tincidunt arcu diam non nulla. Mauris auctor, diam ac cursus scelerisque, diam lorem vulputate urna, ut fringilla turpis nunc eget felis. In hac habitasse platea dictumst. Nunc id leo libero. Sed nunc nunc, venenatis sit amet semper mollis, lacinia non purus. Vestibulum id fermentum diam. Suspendisse faucibus massa ac sapien fermentum quis commodo ipsum tempus. Duis sagittis consectetur mi a tincidunt. In quam eros, viverra quis malesuada a, iaculis ac neque. Etiam pharetra, odio varius pharetra interdum, mauris dolor sodales leo, a sagittis enim orci in nulla. Nunc eu quam ligula, nec iaculis tortor.

Sed suscipit interdum vulputate. Nulla vitae accumsan eros. Etiam eu consectetur enim. Duis et erat ut neque vestibulum malesuada. Cras et lectus dolor. Nullam feugiat semper mattis. Suspendisse non justo aliquet risus mattis tristique nec ut orci. Nulla facilisi. Ut bibendum neque sed nibh egestas a cursus mi volutpat. Morbi ac mi a enim tristique cursus. Pellentesque vitae diam massa. Donec dignissim blandit sollicitudin. Nulla dignissim dolor est, ut dictum libero. Ut sagittis tempus semper. Morbi vitae convallis dui. Quisque ut quam massa.

Nunc bibendum orci id nisl posuere sit amet egestas nisl dictum. Cras faucibus hendrerit elit, non suscipit lacus pharetra in. Vivamus eu libero quis risus laoreet egestas. Morbi eleifend aliquet nisi id semper. Cras consequat est id urna malesuada scelerisque. Aliquam tristique, orci vitae suscipit imperdiet, enim magna laoreet leo, in dignissim mi elit vitae lacus. Morbi ut enim vel enim pretium elementum. Maecenas ultricies blandit nisl, eget tincidunt ligula cursus mattis. Fusce et massa non risus euismod gravida at quis quam. Duis sed nibh justo, id semper justo.

Duis tristique, sapien et posuere iaculis, sapien lectus ultricies risus, nec placerat risus diam varius arcu. Curabitur eget metus id metus placerat dignissim. Curabitur faucibus est erat, at fermentum lorem. Sed interdum imperdiet erat, vitae dapibus eros feugiat non. Donec sit amet massa nec tortor elementum blandit. Vivamus eget magna odio. Donec hendrerit diam scelerisque arcu porta ut tempus lorem vulputate. Mauris sed tellus in risus bibendum luctus eu et sapien. In et scelerisque purus. Ut vel enim dolor. Proin vitae ipsum odio, nec pretium enim.

Nunc id ligula at felis vestibulum accumsan. Sed molestie erat eu nibh semper tincidunt. In iaculis nisl sed sem dictum ut viverra velit tincidunt. Nunc eros elit, dapibus adipiscing iaculis ut, facilisis sit amet arcu. In blandit, libero vel ornare suscipit, orci eros egestas nisi, ac tincidunt arcu diam non nulla. Mauris auctor, diam ac cursus scelerisque, diam lorem vulputate urna, ut fringilla turpis nunc eget felis. In hac habitasse platea dictumst. Nunc id leo libero. Sed nunc nunc, venenatis sit amet semper mollis, lacinia non purus. Vestibulum id fermentum diam. Suspendisse faucibus massa ac sapien fermentum quis commodo ipsum tempus. Duis sagittis consectetur mi a tincidunt. In quam eros, viverra quis malesuada a, iaculis ac neque. Etiam pharetra, odio varius pharetra interdum, mauris dolor sodales leo, a sagittis enim orci in nulla. Nunc eu quam ligula, nec iaculis tortor.

Sed suscipit interdum vulputate. Nulla vitae accumsan eros. Etiam eu consectetur enim. Duis et erat ut neque vestibulum malesuada. Cras et lectus dolor. Nullam feugiat semper mattis. Suspendisse non justo aliquet risus mattis tristique nec ut orci. Nulla facilisi. Ut bibendum neque sed nibh egestas a cursus mi volutpat. Morbi ac mi a enim tristique cursus. Pellentesque vitae diam massa. Donec dignissim blandit sollicitudin. Nulla dignissim dolor est, ut dictum libero. Ut sagittis tempus semper. Morbi vitae convallis dui. Quisque ut quam massa.

Nunc bibendum orci id nisl posuere sit amet egestas nisl dictum. Cras faucibus hendrerit elit, non suscipit lacus pharetra in. Vivamus eu libero quis risus laoreet egestas. Morbi eleifend aliquet nisi id semper. Cras consequat est id urna malesuada scelerisque. Aliquam tristique, orci vitae suscipit imperdiet, enim magna laoreet leo, in dignissim mi elit vitae lacus. Morbi ut enim vel enim pretium elementum. Maecenas ultricies blandit nisl, eget tincidunt ligula cursus mattis. Fusce et massa non risus euismod gravida at quis quam. Duis sed nibh justo, id semper justo.
Duis tristique, sapien et posuere iaculis, sapien lectus ultricies risus, nec placerat risus diam varius arcu. Curabitur eget metus id metus placerat dignissim. Curabitur faucibus est erat, at fermentum lorem. Sed interdum imperdiet erat, vitae dapibus eros feugiat non. Donec sit amet massa nec tortor elementum blandit. Vivamus eget magna odio. Donec hendrerit diam scelerisque arcu porta ut tempus lorem vulputate. Mauris sed tellus in risus bibendum luctus eu et sapien. In et scelerisque purus. Ut vel enim dolor. Proin vitae ipsum odio, nec pretium enim.

Nunc id ligula at felis vestibulum accumsan. Sed molestie erat eu nibh semper tincidunt. In iaculis nisl sed sem dictum ut viverra velit tincidunt. Nunc eros elit, dapibus adipiscing iaculis ut, facilisis sit amet arcu. In blandit, libero vel ornare suscipit, orci eros egestas nisi, ac tincidunt arcu diam non nulla. Mauris auctor, diam ac cursus scelerisque, diam lorem vulputate urna, ut fringilla turpis nunc eget felis. In hac habitasse platea dictumst. Nunc id leo libero. Sed nunc nunc, venenatis sit amet semper mollis, lacinia non purus. Vestibulum id fermentum diam. Suspendisse faucibus massa ac sapien fermentum quis commodo ipsum tempus. Duis sagittis consectetur mi a tincidunt. In quam eros, viverra quis malesuada a, iaculis ac neque. Etiam pharetra, odio varius pharetra interdum, mauris dolor sodales leo, a sagittis enim orci in nulla. Nunc eu quam ligula, nec iaculis tortor.

Sed suscipit interdum vulputate. Nulla vitae accumsan eros. Etiam eu consectetur enim. Duis et erat ut neque vestibulum malesuada. Cras et lectus dolor. Nullam feugiat semper mattis. Suspendisse non justo aliquet risus mattis tristique nec ut orci. Nulla facilisi. Ut bibendum neque sed nibh egestas a cursus mi volutpat. Morbi ac mi a enim tristique cursus. Pellentesque vitae diam massa. Donec dignissim blandit sollicitudin. Nulla dignissim dolor est, ut dictum libero. Ut sagittis tempus semper. Morbi vitae convallis dui. Quisque ut quam massa.

Nunc bibendum orci id nisl posuere sit amet egestas nisl dictum. Cras faucibus hendrerit elit, non suscipit lacus pharetra in. Vivamus eu libero quis risus laoreet egestas. Morbi eleifend aliquet nisi id semper. Cras consequat est id urna malesuada scelerisque. Aliquam tristique, orci vitae suscipit imperdiet, enim magna laoreet leo, in dignissim mi elit vitae lacus. Morbi ut enim vel enim pretium elementum. Maecenas ultricies blandit nisl, eget tincidunt ligula cursus mattis. Fusce et massa non risus euismod gravida at quis quam. Duis sed nibh justo, id semper justo.
Duis tristique, sapien et posuere iaculis, sapien lectus ultricies risus, nec placerat risus diam varius arcu. Curabitur eget metus id metus placerat dignissim. Curabitur faucibus est erat, at fermentum lorem. Sed interdum imperdiet erat, vitae dapibus eros feugiat non. Donec sit amet massa nec tortor elementum blandit. Vivamus eget magna odio. Donec hendrerit diam scelerisque arcu porta ut tempus lorem vulputate. Mauris sed tellus in risus bibendum luctus eu et sapien. In et scelerisque purus. Ut vel enim dolor. Proin vitae ipsum odio, nec pretium enim.

Nunc id ligula at felis vestibulum accumsan. Sed molestie erat eu nibh semper tincidunt. In iaculis nisl sed sem dictum ut viverra velit tincidunt. Nunc eros elit, dapibus adipiscing iaculis ut, facilisis sit amet arcu. In blandit, libero vel ornare suscipit, orci eros egestas nisi, ac tincidunt arcu diam non nulla. Mauris auctor, diam ac cursus scelerisque, diam lorem vulputate urna, ut fringilla turpis nunc eget felis. In hac habitasse platea dictumst. Nunc id leo libero. Sed nunc nunc, venenatis sit amet semper mollis, lacinia non purus. Vestibulum id fermentum diam. Suspendisse faucibus massa ac sapien fermentum quis commodo ipsum tempus. Duis sagittis consectetur mi a tincidunt. In quam eros, viverra quis malesuada a, iaculis ac neque. Etiam pharetra, odio varius pharetra interdum, mauris dolor sodales leo, a sagittis enim orci in nulla. Nunc eu quam ligula, nec iaculis tortor.

Sed suscipit interdum vulputate. Nulla vitae accumsan eros. Etiam eu consectetur enim. Duis et erat ut neque vestibulum malesuada. Cras et lectus dolor. Nullam feugiat semper mattis. Suspendisse non justo aliquet risus mattis tristique nec ut orci. Nulla facilisi. Ut bibendum neque sed nibh egestas a cursus mi volutpat. Morbi ac mi a enim tristique cursus. Pellentesque vitae diam massa. Donec dignissim blandit sollicitudin. Nulla dignissim dolor est, ut dictum libero. Ut sagittis tempus semper. Morbi vitae convallis dui. Quisque ut quam massa.

Nunc bibendum orci id nisl posuere sit amet egestas nisl dictum. Cras faucibus hendrerit elit, non suscipit lacus pharetra in. Vivamus eu libero quis risus laoreet egestas. Morbi eleifend aliquet nisi id semper. Cras consequat est id urna malesuada scelerisque. Aliquam tristique, orci vitae suscipit imperdiet, enim magna laoreet leo, in dignissim mi elit vitae lacus. Morbi ut enim vel enim pretium elementum. Maecenas ultricies blandit nisl, eget tincidunt ligula cursus mattis. Fusce et massa non risus euismod gravida at quis quam. Duis sed nibh justo, id semper justo.
Duis tristique, sapien et posuere iaculis, sapien lectus ultricies risus, nec placerat risus diam varius arcu. Curabitur eget metus id metus placerat dignissim. Curabitur faucibus est erat, at fermentum lorem. Sed interdum imperdiet erat, vitae dapibus eros feugiat non. Donec sit amet massa nec tortor elementum blandit. Vivamus eget magna odio. Donec hendrerit diam scelerisque arcu porta ut tempus lorem vulputate. Mauris sed tellus in risus bibendum luctus eu et sapien. In et scelerisque purus. Ut vel enim dolor. Proin vitae ipsum odio, nec pretium enim.

Nunc id ligula at felis vestibulum accumsan. Sed molestie erat eu nibh semper tincidunt. In iaculis nisl sed sem dictum ut viverra velit tincidunt. Nunc eros elit, dapibus adipiscing iaculis ut, facilisis sit amet arcu. In blandit, libero vel ornare suscipit, orci eros egestas nisi, ac tincidunt arcu diam non nulla. Mauris auctor, diam ac cursus scelerisque, diam lorem vulputate urna, ut fringilla turpis nunc eget felis. In hac habitasse platea dictumst. Nunc id leo libero. Sed nunc nunc, venenatis sit amet semper mollis, lacinia non purus. Vestibulum id fermentum diam. Suspendisse faucibus massa ac sapien fermentum quis commodo ipsum tempus. Duis sagittis consectetur mi a tincidunt. In quam eros, viverra quis malesuada a, iaculis ac neque. Etiam pharetra, odio varius pharetra interdum, mauris dolor sodales leo, a sagittis enim orci in nulla. Nunc eu quam ligula, nec iaculis tortor.

Sed suscipit interdum vulputate. Nulla vitae accumsan eros. Etiam eu consectetur enim. Duis et erat ut neque vestibulum malesuada. Cras et lectus dolor. Nullam feugiat semper mattis. Suspendisse non justo aliquet risus mattis tristique nec ut orci. Nulla facilisi. Ut bibendum neque sed nibh egestas a cursus mi volutpat. Morbi ac mi a enim tristique cursus. Pellentesque vitae diam massa. Donec dignissim blandit sollicitudin. Nulla dignissim dolor est, ut dictum libero. Ut sagittis tempus semper. Morbi vitae convallis dui. Quisque ut quam massa.

Nunc bibendum orci id nisl posuere sit amet egestas nisl dictum. Cras faucibus hendrerit elit, non suscipit lacus pharetra in. Vivamus eu libero quis risus laoreet egestas. Morbi eleifend aliquet nisi id semper. Cras consequat est id urna malesuada scelerisque. Aliquam tristique, orci vitae suscipit imperdiet, enim magna laoreet leo, in dignissim mi elit vitae lacus. Morbi ut enim vel enim pretium elementum. Maecenas ultricies blandit nisl, eget tincidunt ligula cursus mattis. Fusce et massa non risus euismod gravida at quis quam. Duis sed nibh justo, id semper justo.

%%%%%%%%%%%%%%%%%%%%%%%%%%%%%%%%%%%%%%%%%%%%%%%%%%%%%%%%%%%%
\chapter{Drugi rozdzia�}
\noindent
	Wst�p Lorem \cite{bramwell2001} ipsum dolor sit amet, consectetur adipiscing elit. Integer odio lectus, egestas sed aliquet ut, semper et tortor. Ut consectetur augue non\cite{padfield2007}metus scelerisque accumsan. Ut quis odio dolor. Suspendisse eu ante ut lorem sagittis lacinia. Mauris elit neque, eleifend eget varius ut, adipiscing sit amet quam. Duis sit amet ullamcorper purus. Aliquam erat volutpat. Donec pharetra massa sed mauris vehicula aliquet eget eu nulla. Etiam non felis metus. Etiam consequat commodo ante. Cras accumsan risus eu enim volutpat quis lobortis risus sagittis. Mauris posuere rutrum ante, at consectetur massa iaculis in.

Duis tristique, sapien et posuere iaculis, sapien lectus ultricies risus, nec placerat risus diam varius arcu. Curabitur eget metus id metus placerat dignissim. Curabitur faucibus est erat, at fermentum lorem. Sed interdum imperdiet erat, vitae dapibus eros feugiat non. Donec sit amet massa nec tortor elementum blandit. Vivamus eget magna odio. Donec hendrerit diam scelerisque arcu porta ut tempus lorem vulputate. Mauris sed tellus in risus bibendum luctus eu et sapien. In et scelerisque purus. Ut vel enim dolor. Proin vitae ipsum odio, nec pretium enim.

Nunc id ligula at felis vestibulum accumsan. Sed molestie erat eu nibh semper tincidunt. In iaculis nisl sed sem dictum ut viverra velit tincidunt. Nunc eros elit, dapibus adipiscing iaculis ut, facilisis sit amet arcu. In blandit, libero vel ornare suscipit, orci eros egestas nisi, ac tincidunt arcu diam non nulla. Mauris auctor, diam ac cursus scelerisque, diam lorem vulputate urna, ut fringilla turpis nunc eget felis. In hac habitasse platea dictumst. Nunc id leo libero. Sed nunc nunc, venenatis sit amet semper mollis, lacinia non purus. Vestibulum id fermentum diam. Suspendisse faucibus massa ac sapien fermentum quis commodo ipsum tempus. Duis sagittis consectetur mi a tincidunt. In quam eros, viverra quis malesuada a, iaculis ac neque. Etiam pharetra, odio varius pharetra interdum, mauris dolor sodales leo, a sagittis enim orci in nulla. Nunc eu quam ligula, nec iaculis tortor.

Sed suscipit interdum vulputate. Nulla vitae accumsan eros. Etiam eu consectetur enim. Duis et erat ut neque vestibulum malesuada. Cras et lectus dolor. Nullam feugiat semper mattis. Suspendisse non justo aliquet risus mattis tristique nec ut orci. Nulla facilisi. Ut bibendum neque sed nibh egestas a cursus mi volutpat. Morbi ac mi a enim tristique cursus. Pellentesque vitae diam massa. Donec dignissim blandit sollicitudin. Nulla dignissim dolor est, ut dictum libero. Ut sagittis tempus semper. Morbi vitae convallis dui. Quisque ut quam massa.

Nunc bibendum orci id nisl posuere sit amet egestas nisl dictum. Cras faucibus hendrerit elit, non suscipit lacus pharetra in. Vivamus eu libero quis risus laoreet egestas. Morbi eleifend aliquet nisi id semper. Cras consequat est id urna malesuada scelerisque. Aliquam tristique, orci vitae suscipit imperdiet, enim magna laoreet leo, in dignissim mi elit vitae lacus. Morbi ut enim vel enim pretium elementum. Maecenas ultricies blandit nisl, eget tincidunt ligula cursus mattis. Fusce et massa non risus euismod gravida at quis quam. Duis sed nibh justo, id semper justo.

Duis tristique, sapien et posuere iaculis, sapien lectus ultricies risus, nec placerat risus diam varius arcu. Curabitur eget metus id metus placerat dignissim. Curabitur faucibus est erat, at fermentum lorem. Sed interdum imperdiet erat, vitae dapibus eros feugiat non. Donec sit amet massa nec tortor elementum blandit. Vivamus eget magna odio. Donec hendrerit diam scelerisque arcu porta ut tempus lorem vulputate. Mauris sed tellus in risus bibendum luctus eu et sapien. In et scelerisque purus. Ut vel enim dolor. Proin vitae ipsum odio, nec pretium enim.

Nunc id ligula at felis vestibulum accumsan. Sed molestie erat eu nibh semper tincidunt. In iaculis nisl sed sem dictum ut viverra velit tincidunt. Nunc eros elit, dapibus adipiscing iaculis ut, facilisis sit amet arcu. In blandit, libero vel ornare suscipit, orci eros egestas nisi, ac tincidunt arcu diam non nulla. Mauris auctor, diam ac cursus scelerisque, diam lorem vulputate urna, ut fringilla turpis nunc eget felis. In hac habitasse platea dictumst. Nunc id leo libero. Sed nunc nunc, venenatis sit amet semper mollis, lacinia non purus. Vestibulum id fermentum diam. Suspendisse faucibus massa ac sapien fermentum quis commodo ipsum tempus. Duis sagittis consectetur mi a tincidunt. In quam eros, viverra quis malesuada a, iaculis ac neque. Etiam pharetra, odio varius pharetra interdum, mauris dolor sodales leo, a sagittis enim orci in nulla. Nunc eu quam ligula, nec iaculis tortor.

Sed suscipit interdum vulputate. Nulla vitae accumsan eros. Etiam eu consectetur enim. Duis et erat ut neque vestibulum malesuada. Cras et lectus dolor. Nullam feugiat semper mattis. Suspendisse non justo aliquet risus mattis tristique nec ut orci. Nulla facilisi. Ut bibendum neque sed nibh egestas a cursus mi volutpat. Morbi ac mi a enim tristique cursus. Pellentesque vitae diam massa. Donec dignissim blandit sollicitudin. Nulla dignissim dolor est, ut dictum libero. Ut sagittis tempus semper. Morbi vitae convallis dui. Quisque ut quam massa.

Nunc bibendum orci id nisl posuere sit amet egestas nisl dictum. Cras faucibus hendrerit elit, non suscipit lacus pharetra in. Vivamus eu libero quis risus laoreet egestas. Morbi eleifend aliquet nisi id semper. Cras consequat est id urna malesuada scelerisque. Aliquam tristique, orci vitae suscipit imperdiet, enim magna laoreet leo, in dignissim mi elit vitae lacus. Morbi ut enim vel enim pretium elementum. Maecenas ultricies blandit nisl, eget tincidunt ligula cursus mattis. Fusce et massa non risus euismod gravida at quis quam. Duis sed nibh justo, id semper justo.

Duis tristique, sapien et posuere iaculis, sapien lectus ultricies risus, nec placerat risus diam varius arcu. Curabitur eget metus id metus placerat dignissim. Curabitur faucibus est erat, at fermentum lorem. Sed interdum imperdiet erat, vitae dapibus eros feugiat non. Donec sit amet massa nec tortor elementum blandit. Vivamus eget magna odio. Donec hendrerit diam scelerisque arcu porta ut tempus lorem vulputate. Mauris sed tellus in risus bibendum luctus eu et sapien. In et scelerisque purus. Ut vel enim dolor. Proin vitae ipsum odio, nec pretium enim.

Nunc id ligula at felis vestibulum accumsan. Sed molestie erat eu nibh semper tincidunt. In iaculis nisl sed sem dictum ut viverra velit tincidunt. Nunc eros elit, dapibus adipiscing iaculis ut, facilisis sit amet arcu. In blandit, libero vel ornare suscipit, orci eros egestas nisi, ac tincidunt arcu diam non nulla. Mauris auctor, diam ac cursus scelerisque, diam lorem vulputate urna, ut fringilla turpis nunc eget felis. In hac habitasse platea dictumst. Nunc id leo libero. Sed nunc nunc, venenatis sit amet semper mollis, lacinia non purus. Vestibulum id fermentum diam. Suspendisse faucibus massa ac sapien fermentum quis commodo ipsum tempus. Duis sagittis consectetur mi a tincidunt. In quam eros, viverra quis malesuada a, iaculis ac neque. Etiam pharetra, odio varius pharetra interdum, mauris dolor sodales leo, a sagittis enim orci in nulla. Nunc eu quam ligula, nec iaculis tortor.

Sed suscipit interdum vulputate. Nulla vitae accumsan eros. Etiam eu consectetur enim. Duis et erat ut neque vestibulum malesuada. Cras et lectus dolor. Nullam feugiat semper mattis. Suspendisse non justo aliquet risus mattis tristique nec ut orci. Nulla facilisi. Ut bibendum neque sed nibh egestas a cursus mi volutpat. Morbi ac mi a enim tristique cursus. Pellentesque vitae diam massa. Donec dignissim blandit sollicitudin. Nulla dignissim dolor est, ut dictum libero. Ut sagittis tempus semper. Morbi vitae convallis dui. Quisque ut quam massa.

Nunc bibendum orci id nisl posuere sit amet egestas nisl dictum. Cras faucibus hendrerit elit, non suscipit lacus pharetra in. Vivamus eu libero quis risus laoreet egestas. Morbi eleifend aliquet nisi id semper. Cras consequat est id urna malesuada scelerisque. Aliquam tristique, orci vitae suscipit imperdiet, enim magna laoreet leo, in dignissim mi elit vitae lacus. Morbi ut enim vel enim pretium elementum. Maecenas ultricies blandit nisl, eget tincidunt ligula cursus mattis. Fusce et massa non risus euismod gravida at quis quam. Duis sed nibh justo, id semper justo.
Duis tristique, sapien et posuere iaculis, sapien lectus ultricies risus, nec placerat risus diam varius arcu. Curabitur eget metus id metus placerat dignissim. Curabitur faucibus est erat, at fermentum lorem. Sed interdum imperdiet erat, vitae dapibus eros feugiat non. Donec sit amet massa nec tortor elementum blandit. Vivamus eget magna odio. Donec hendrerit diam scelerisque arcu porta ut tempus lorem vulputate. Mauris sed tellus in risus bibendum luctus eu et sapien. In et scelerisque purus. Ut vel enim dolor. Proin vitae ipsum odio, nec pretium enim.

Nunc id ligula at felis vestibulum accumsan. Sed molestie erat eu nibh semper tincidunt. In iaculis nisl sed sem dictum ut viverra velit tincidunt. Nunc eros elit, dapibus adipiscing iaculis ut, facilisis sit amet arcu. In blandit, libero vel ornare suscipit, orci eros egestas nisi, ac tincidunt arcu diam non nulla. Mauris auctor, diam ac cursus scelerisque, diam lorem vulputate urna, ut fringilla turpis nunc eget felis. In hac habitasse platea dictumst. Nunc id leo libero. Sed nunc nunc, venenatis sit amet semper mollis, lacinia non purus. Vestibulum id fermentum diam. Suspendisse faucibus massa ac sapien fermentum quis commodo ipsum tempus. Duis sagittis consectetur mi a tincidunt. In quam eros, viverra quis malesuada a, iaculis ac neque. Etiam pharetra, odio varius pharetra interdum, mauris dolor sodales leo, a sagittis enim orci in nulla. Nunc eu quam ligula, nec iaculis tortor.

Sed suscipit interdum vulputate. Nulla vitae accumsan eros. Etiam eu consectetur enim. Duis et erat ut neque vestibulum malesuada. Cras et lectus dolor. Nullam feugiat semper mattis. Suspendisse non justo aliquet risus mattis tristique nec ut orci. Nulla facilisi. Ut bibendum neque sed nibh egestas a cursus mi volutpat. Morbi ac mi a enim tristique cursus. Pellentesque vitae diam massa. Donec dignissim blandit sollicitudin. Nulla dignissim dolor est, ut dictum libero. Ut sagittis tempus semper. Morbi vitae convallis dui. Quisque ut quam massa.

Nunc bibendum orci id nisl posuere sit amet egestas nisl dictum. Cras faucibus hendrerit elit, non suscipit lacus pharetra in. Vivamus eu libero quis risus laoreet egestas. Morbi eleifend aliquet nisi id semper. Cras consequat est id urna malesuada scelerisque. Aliquam tristique, orci vitae suscipit imperdiet, enim magna laoreet leo, in dignissim mi elit vitae lacus. Morbi ut enim vel enim pretium elementum. Maecenas ultricies blandit nisl, eget tincidunt ligula cursus mattis. Fusce et massa non risus euismod gravida at quis quam. Duis sed nibh justo, id semper justo.
Duis tristique, sapien et posuere iaculis, sapien lectus ultricies risus, nec placerat risus diam varius arcu. Curabitur eget metus id metus placerat dignissim. Curabitur faucibus est erat, at fermentum lorem. Sed interdum imperdiet erat, vitae dapibus eros feugiat non. Donec sit amet massa nec tortor elementum blandit. Vivamus eget magna odio. Donec hendrerit diam scelerisque arcu porta ut tempus lorem vulputate. Mauris sed tellus in risus bibendum luctus eu et sapien. In et scelerisque purus. Ut vel enim dolor. Proin vitae ipsum odio, nec pretium enim.

Nunc id ligula at felis vestibulum accumsan. Sed molestie erat eu nibh semper tincidunt. In iaculis nisl sed sem dictum ut viverra velit tincidunt. Nunc eros elit, dapibus adipiscing iaculis ut, facilisis sit amet arcu. In blandit, libero vel ornare suscipit, orci eros egestas nisi, ac tincidunt arcu diam non nulla. Mauris auctor, diam ac cursus scelerisque, diam lorem vulputate urna, ut fringilla turpis nunc eget felis. In hac habitasse platea dictumst. Nunc id leo libero. Sed nunc nunc, venenatis sit amet semper mollis, lacinia non purus. Vestibulum id fermentum diam. Suspendisse faucibus massa ac sapien fermentum quis commodo ipsum tempus. Duis sagittis consectetur mi a tincidunt. In quam eros, viverra quis malesuada a, iaculis ac neque. Etiam pharetra, odio varius pharetra interdum, mauris dolor sodales leo, a sagittis enim orci in nulla. Nunc eu quam ligula, nec iaculis tortor.

Sed suscipit interdum vulputate. Nulla vitae accumsan eros. Etiam eu consectetur enim. Duis et erat ut neque vestibulum malesuada. Cras et lectus dolor. Nullam feugiat semper mattis. Suspendisse non justo aliquet risus mattis tristique nec ut orci. Nulla facilisi. Ut bibendum neque sed nibh egestas a cursus mi volutpat. Morbi ac mi a enim tristique cursus. Pellentesque vitae diam massa. Donec dignissim blandit sollicitudin. Nulla dignissim dolor est, ut dictum libero. Ut sagittis tempus semper. Morbi vitae convallis dui. Quisque ut quam massa.

Nunc bibendum orci id nisl posuere sit amet egestas nisl dictum. Cras faucibus hendrerit elit, non suscipit lacus pharetra in. Vivamus eu libero quis risus laoreet egestas. Morbi eleifend aliquet nisi id semper. Cras consequat est id urna malesuada scelerisque. Aliquam tristique, orci vitae suscipit imperdiet, enim magna laoreet leo, in dignissim mi elit vitae lacus. Morbi ut enim vel enim pretium elementum. Maecenas ultricies blandit nisl, eget tincidunt ligula cursus mattis. Fusce et massa non risus euismod gravida at quis quam. Duis sed nibh justo, id semper justo.
Duis tristique, sapien et posuere iaculis, sapien lectus ultricies risus, nec placerat risus diam varius arcu. Curabitur eget metus id metus placerat dignissim. Curabitur faucibus est erat, at fermentum lorem. Sed interdum imperdiet erat, vitae dapibus eros feugiat non. Donec sit amet massa nec tortor elementum blandit. Vivamus eget magna odio. Donec hendrerit diam scelerisque arcu porta ut tempus lorem vulputate. Mauris sed tellus in risus bibendum luctus eu et sapien. In et scelerisque purus. Ut vel enim dolor. Proin vitae ipsum odio, nec pretium enim.

Nunc id ligula at felis vestibulum accumsan. Sed molestie erat eu nibh semper tincidunt. In iaculis nisl sed sem dictum ut viverra velit tincidunt. Nunc eros elit, dapibus adipiscing iaculis ut, facilisis sit amet arcu. In blandit, libero vel ornare suscipit, orci eros egestas nisi, ac tincidunt arcu diam non nulla. Mauris auctor, diam ac cursus scelerisque, diam lorem vulputate urna, ut fringilla turpis nunc eget felis. In hac habitasse platea dictumst. Nunc id leo libero. Sed nunc nunc, venenatis sit amet semper mollis, lacinia non purus. Vestibulum id fermentum diam. Suspendisse faucibus massa ac sapien fermentum quis commodo ipsum tempus. Duis sagittis consectetur mi a tincidunt. In quam eros, viverra quis malesuada a, iaculis ac neque. Etiam pharetra, odio varius pharetra interdum, mauris dolor sodales leo, a sagittis enim orci in nulla. Nunc eu quam ligula, nec iaculis tortor.

Sed suscipit interdum vulputate. Nulla vitae accumsan eros. Etiam eu consectetur enim. Duis et erat ut neque vestibulum malesuada. Cras et lectus dolor. Nullam feugiat semper mattis. Suspendisse non justo aliquet risus mattis tristique nec ut orci. Nulla facilisi. Ut bibendum neque sed nibh egestas a cursus mi volutpat. Morbi ac mi a enim tristique cursus. Pellentesque vitae diam massa. Donec dignissim blandit sollicitudin. Nulla dignissim dolor est, ut dictum libero. Ut sagittis tempus semper. Morbi vitae convallis dui. Quisque ut quam massa.

Nunc bibendum orci id nisl posuere sit amet egestas nisl dictum. Cras faucibus hendrerit elit, non suscipit lacus pharetra in. Vivamus eu libero quis risus laoreet egestas. Morbi eleifend aliquet nisi id semper. Cras consequat est id urna malesuada scelerisque. Aliquam tristique, orci vitae suscipit imperdiet, enim magna laoreet leo, in dignissim mi elit vitae lacus. Morbi ut enim vel enim pretium elementum. Maecenas ultricies blandit nisl, eget tincidunt ligula cursus mattis. Fusce et massa non risus euismod gravida at quis quam. Duis sed nibh justo, id semper justo.

%%%%%%%%%%%%%%%%%%%%%%%%%%%%%%%%%%%%%%%%%%%%%%%%%%%%%%%%%%%%
 
\listoffigures
\listoftables

\bibliographystyle{iisthesis}
\bibliography{bibliografia}

\appendix
\chapter{Co� dodatkowego}
\pagestyle{plain}

\end{document}
