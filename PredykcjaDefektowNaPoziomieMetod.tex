\documentclass[twoside]{iisthesis}
\usepackage[MeX]{polski}
\usepackage[cp1250]{inputenc}
\usepackage{graphicx}
\usepackage{cite}
\usepackage{xcolor}
\usepackage{epstopdf}
\usepackage{tablefootnote}
\usepackage{siunitx}
\usepackage{booktabs}

% definicje kolorow
\definecolor{ciemnoSzary}{rgb}{0.15,0.15,0.15}
\definecolor{szary}{rgb}{0.5,0.5,0.5}
\definecolor{jasnoSzary}{rgb}{0.2,0.2,0.2}
\newcommand\todo[1]{\textcolor{red}{#1}}

\begin{document}

% Zmiana domy�lnych angielskich nazw cz�ci dokumentu na polskie "Rozdzia�y s� w klasie issthesis"
% wst�pnie poprawione mo�e kto� b�dzie zna� lepszy spos�b i wrzuci to do klasy issthesis.cls
\renewcommand{\contentsname}{Spis tre�ci}
\renewcommand{\appendixname}{Dodatek}
\renewcommand{\listfigurename}{Spis ilustracji}
\renewcommand{\listtablename}{Spis tabel}
\renewcommand{\refname}{Bibliografia}
\renewcommand{\abstractname}{Streszczenie}

\title{Predykcja defekt�w na poziomie metod w celu zredukowania wysi�ku zwi�zanego z zapewnieniem jako�ci oprogramowania}
\author{Mateusz Kutyba}
\advisor{dr hab. in�. Lech Madeyski}
\instituteLogo{logos/pwr}
\slowaKluczowe{predykcja defekt�w\\metryki oprogramowania\\oceny oparte na wysi�ku}

\date{\number\the\year}

% Wstawienie streszczenia pracy
\abstractSH{Przewidywanie b��d�w w oprogramowaniu z wykorzystaniem algorytm�w uczenia maszynowego, na podstawie metryk oprogramowania i danych historycznych. \newline \newline}
\abstractPL{Streszczenie po polsku}
\abstractEN{Abctract in english}

%spis tre�ci
\maketitle
\textpages

%%%%%%%%%%%%%%%%%%%%%%%%%%%%%%%%%%%%%%%%%%%%%%%%%%%%%%%%%%%%
\chapter{Wst�p}
\label{rozdzial1}
\noindent
Prawdopodobnie nie istniej� programy wolne od b��d�w. Z~ca�� pewno�ci� istniej� programy, kt�re zawieraj� zbyt du�� liczb� b��d�w. Ka�dy kto tworzy oprogramowanie chcia�by aby by�o ono wolne od wad. Podstawowym narz�dziem pozwalaj�cym na sprawdzenie czy program dzia�a poprawnie s� testy. Powstaj� coraz bardziej wyszukane metody i~metodyki testowania oprogramowania, a~wszystko po to aby oprogramowanie dzia�a�o zgodnie z~oczekiwaniami, czyli aby cechowa�o si� wysok� jako�ci�. Testowanie i~inspekcje kodu pozwalaj� zapewni� odpowiedni� jako�� oprogramowania, ale s� kosztowne. Bazuj�c na zasadzie Pareto \cite{endres2003handbook} wiemy, �e oko�o 80\% defekt�w pochodzi z~20\% modu��w \cite{concas2011distribution}. Wiedz�c, kt�re modu�y nale�y podda� inspekcji, mo�na znacznie obni�y� ilo�� pracy potrzebn� do znalezienia wi�kszo�ci b��d�w, a~co za tym idzie znacz�co zmniejszy� koszt takich inspekcji.

Kod �r�d�owy oprogramowania zazwyczaj sk�ada si� z~wielu plik�w, kt�re s� organizowane w~pakiety (ang. \textit{packages}). W~j�zyku Java pliki zawieraj� klasy (ang. \textit{class}), a~ka�da klasa mo�e zawiera� metody (ang. \textit{method}). Powsta�o wiele metod predykcji defekt�w, na r�nych poziowach granulacji: od pakiet�w, przez pliki, klasy, metody, na pojedynczych zmianach (ang. \textit{hunk} \cite{ferzund2009empirical, ferzund2009software}) sko�czywszy \cite{hall2012systematic, catal2009systematic, d2010extensive, d2012evaluating}. Jak wykazano m.in. w~\cite{giger2012method} i~\cite{hata2012bug}, predykcja b��d�w na poziomie metod dostarcza dok�adniejszych danych na temat lokalizacji b��d�w, dzi�ki czemu ich odnajdywanie jest efektywniejsze.

Predykcja defekt�w oprogramowania wykorzystuje techniki eksploracji danych, g��wnie s� to metody statystyczne i~metody uczenia maszynowego. Kluczowym elementem w~tych procesach s� w�a�nie dane. To na ich podstawie algorytmy uczenia maszynowego s� w~stanie formu�owa� regu�y decyzyjne. W~in�ynierii oprogramowania tymi danymi s� r�znego rodzaju metryki oprogramowania. Podzia� metryk oraz ich zastosowanie zosta�y szerzej opisane w~rozdziale \ref{rola-metryk}.

Gromadzenie danych (metryk) z~projekt�w jest czasoch�onne, wymaga du�o pracy --- pobierania lub kopiowania projekt�w, mocy obliczeniowej do wyliczenia metryk. Potrzebne jest stworzenie uniwersalnych rozwi�za� s�u��cych do tego celu oraz nastawienie na mo�liwo�� rozszerzania zestawu narz�dzi, kt�re mog� by� ze sob� dowolnie zestawiane. Te wymagania spe�nia platforma DePress \cite{madeyski2014software}, kt�ra jest rozwijana przy udziale student�w i~pracownik�w Politechniki Wroc�awskiej oraz pracownik�w Capgemini Polska. Wi�cej informacji o~DePress zawarto w~rozdziale \ref{depress}.

Wst�pne przeszukiwanie literatury wykaza�o niewielk� liczb� �r�de� �ci�le odpowiadaj�cych zagadnieniu predykcji defekt�w na niskim poziomie granulacji. Jest to g��wny kierunek tych bada� a~ich celem jest przede wszystkim opracowanie nowego modelu, kt�ry mia�by s�u�y� do efektywnego wskazywania miejsc w~oprogramowaniu, w~kt�rych znajduj� si� b��dy. Pozwololi�oby to na ograniczenie ilo�ci pracy potrzebnej do przejrzenia krytycznych miejsc i~naprawienia b��d�w.





%%%%%%%%%%%%%%%%%%%%%%%%%%%%%%%%%%%%%%%%%%%%%%%%%%%%%%%%%%%%
%%%%%%%%%%%%%%%%%%%%%%%%%%%%%%%%%%%%%%%%%%%%%%%%%%%%%%%%%%%%
\section{Cele pracy}

\noindent Cele pracy dyplomowej:
\begin{itemize}
	\item Przegl�d literatury pod k�tem predykcji defekt�w oprogramowania, szczeg�lnie na niskim poziomie granulacji.
	\item Budowa nowych lub rozbudowa istniej�cych narz�dzi s�u��cych do wyliczenia metryk oprogramowania, wsp�pracuj�cych z~wersjonowanymi repozytoriami kodu (wsparcie dla Git).
	\item Zebranie danych z~projekt�w o~otwartych �r�d�ach na potrzeby predykcji defekt�w.
	\item Budowa modeli predykcji z~wykorzystaniem zebranych danych.
	\item Ocena stworzonych narz�dzi oraz zebranych danych.
	\item Ewaluacja modeli predykcji i~ocena ich skuteczno�ci.
\end{itemize}





%%%%%%%%%%%%%%%%%%%%%%%%%%%%%%%%%%%%%%%%%%%%%%%%%%%%%%%%%%%%
\subsection{Przeznaczenie narz�dzi}

\noindent Narz�dzia stworzone w~ramach pracy dyplomowej wchodz� w~sk�ad platformy (ang. \textit{framework}) DePress\footnote{http://depress.io} (\textit{Defect Prediction in Software Systems})\cite{madeyski2014software}. Jest to rozszerzalna platforma pozwalaj�ca na budowanie przep�ywu pracy (ang. \textit{workflow}) w~spos�b graficzny, dzi�ki temu, �e jest oparta na projekcie KNIME. G��wnym celem DePress jest wspieranie analizy empirycznej oprogramowania. Pozwala na zbieranie, ��czenie i~analiz� danych z~r�nych �r�de�, jak repozytoria oprogramowania czy metryki.






%%%%%%%%%%%%%%%%%%%%%%%%%%%%%%%%%%%%%%%%%%%%%%%%%%%%%%%%%%%%
\subsection{Ograniczenia dotycz�ce realizacji}

\noindent Poni�ej wypisano ograniczenia dotycz�ce realizacji bada�:
\begin{itemize}
	\item badanie tylko projekt�w o~otwartych �r�d�ach (ang. \textit{open source}),
	\item badanie tylko projekt�w napisanych w~j�zyku Java,
	\item wykorzystanie narz�dzi platformy DePress, lub stworzenie nowych narz�dzi w~ramach platformy,
	\item mo�liwo�� �atwego, najlepiej zautomatyzowanego powt�rzenia badania,
	\item wykorzystanie j�zyka R~do budowy modeli predykcji defekt�w.
\end{itemize}








%%%%%%%%%%%%%%%%%%%%%%%%%%%%%%%%%%%%%%%%%%%%%%%%%%%%%%%%%%%%
\subsection{Metoda oceny}

\noindent Podstawow� ocen� efektywno�ci tworzonego modelu by�a ocena oparta na wysi�ku (ang. \textit{effort-based evaluation}). Dla ka�dego projektu, kt�ry dostarcza� danych wej�ciowych, zosta�a stworzona krzywa efektywno�ci. Przyk�ad takiej krzywej przedstawia rysunek \ref{example-chart}. Por�wnanie efektywno�ci modelu polega przede wszystkim na por�wnaniu procentowej liczby b��d�w znalezionych w~okre�lonej ilo�ci kodu. Przyj�to, �� warto�ci� graniczn� kodu poddanego inspekcji b�dzie 20\%. Taka sama warto�� jest stosowana w~innych badaniach, m.in. \cite{arisholm2010systematic}, \cite{hata2012bug}, \cite{kamei2010revisiting}, \cite{mende2010effort}, \cite{rahman2011bugcache}. W~przedstawionym przyk�adzie (rys. \ref{example-chart}) dokonuj�c przegl�du 20\% kodu, znajdzie si� w~nim 40\% encji z~defektami (w~przypadku tych bada� s� to metody).

\begin{figure}[htbp]
	\caption{Wykres krzywej efektywno�ci}
	\label{example-chart}
	\centering
	\includegraphics[width=.75\textwidth]{charts/example}
\end{figure}

Dodatkowo zastosowano inne miary skuteczno�ci klasyfikacji, w~nawiasie zawarto symbol kt�rym s� oznaczane:
\begin{itemize}
	\item dok�adno�� ($A$),
	\item wsp�czynnik Kappa Cohena ($\kappa$),
	\item powierzchnia pod krzyw� ROC ($AUC$).
\end{itemize}

\paragraph{Dok�adno��,} ang. \textit{Accuracy}, $A$. Jest to stosunek poprawnie sklasyfikowanych instancji do wszystkich instancji. Maksymalna dok�adno�� wynosz�ca 1 oznacza ca�kowit� zgodno�� wyniku predykcji z~rzeczywistymi klasami. Minimalna warto�� to 0.
\begin{equation}
	A=\frac{TP+TN}{TP+TN+FP+FN}
\end{equation}

\paragraph{Wsp�czynnik Kappa Cohena,} $\kappa$. Jest miar� statystyczn� okre�laj�c� zgodno�� pomi�dzy r�nymi klasyfikatorami. Bierze pod uwag� przypadkow� zgodno��, dzi�ki czemu mo�na okre�li� czy dok�adno�� przewy�sza poziom losowej dok�adno�ci. Wsp�czynnik sprawdza si� dobrze w~problemach gdzie liczno�� instancji w~poszczeg�lnych klasach nie jest r�wna. Maksymalna warto�� Kappa to 1 a~minimalna to -1.
\begin{equation}
	\kappa=\frac{A-RA}{1-RA} \qquad \textrm{gdzie } RA=\frac{(TP+FN)(TP+FP)+(TN+FP)(TN+FN)}{(TP+TN+FP+FN)^{2}}
\end{equation}

\paragraph{Powierzchnia pod krzyw� ROC,} ang. \textit{Area Under the Curve}, $AUC$. Klasyfikatory nie okre�laj� samej przynale�no�ci do klasy, ale warto�� prawdopodobie�stwa z~jakim dana instancja nale�y do danej klasy. Daje to mo�liwo�� wykre�lenia krzywej zale�no�ci pomi�dzy TP i~FP. Krzywa na wykresie okre�lana jest jako ROC (ang. \textit{Receiver Operating Characteristic}). Pole pod t� krzyw� reprezentuje skuteczno�� klasyfikatora. Idealny klasyfikator uzyska wynik 1, natomiast losowy klasyfikator powinien uzyska� wynik 0,5. Wykres \ref{example-roc} przedstawia przyk�ad krzywej ROC.
\begin{figure}[htbp]
	\caption{Wykres krzywej ROC}
	\label{example-roc}
	\centering
	\includegraphics[width=.5\textwidth]{charts/example-roc}
\end{figure}








%%%%%%%%%%%%%%%%%%%%%%%%%%%%%%%%%%%%%%%%%%%%%%%%%%%%%%%%%%%%
%%%%%%%%%%%%%%%%%%%%%%%%%%%%%%%%%%%%%%%%%%%%%%%%%%%%%%%%%%%%
\section{Struktura pracy}

\noindent Dalsza cz�� pracy zosta�a podzielona w~nast�puj�cy spos�b. W~rozdziale \ref{rozdzial2} opisano przegl�d literatury oraz om�wiono aktualny stan wiedzy. W~rozdziale \ref{rozdzial3} zawarto charakterystyk� wykorzystanych narz�dzi, oprogramowania i~j�zyk�w. W~rozdziale \ref{rozdzial4} opisano przebieg bada� i~ich wyniki, natomiast w~rozdziale \ref{rozdzial5} przeanalizowano uzyskane rezultaty, podsumowano badanie pod k�tem jego potencjalnego zastosowania i~mo�liwo�ci dalszego rozwoju. Rozdzia� ten zawiera r�wnie� istotny fragment dotycz�cy zagro�e� dla wiarygodno�ci przeprowadzonego badania.


%%%%%%%%%%%%%%%%%%%%%%%%%%%%%%%%%%%%%%%%%%%%%%%%%%%%%%%%%%%%
\chapter{Przegl�d literatury}
\label{rozdzial2}
\noindent
%%%%%%%%%%%%%%%%%%%%%%%%%%%%%%%%%%%%%%%%%%%%%%%%%%%%%%%%%%%%
%%%%%%%%%%%%%%%%%%%%%%%%%%%%%%%%%%%%%%%%%%%%%%%%%%%%%%%%%%%%
\section{Zwi�zek z~innymi pracami}

\noindent Na pocz�tku prac dokonano przegl�du literatury aby okre�li� aktualny stan wiedzy (ang. \textit{state of the art}) w~badanej dziedzinie. Przegl�d literatury pozwoli� udzieli� odpowiedzi na nast�puj�ce pytania:
\begin{itemize}
	\item {\bf Jakie istniej� metody predykcji defekt�w na poziomie metod i~jaka jest ich skuteczno��?}
		\\Istnieje wiele modeli predykcji defekt�w, jednak wi�kszo�� z~nich opiera si� na danych dotycz�cych klas, pakiet�w lub modu��w. Odpowiedzi� na powy�sze pytanie jest zbi�r modeli predykcji defekt�w na niskim poziomie granulacji, na przyk�ad metod lub blok�w kodu.
	\item {\bf Jakie s� mo�liwo�ci usprawnienia lub rozwini�cia istniej�cych metod?}
		\\Zosta�y zebrane wszelkie mo�liwo�ci ulepszenia lub rozszerzenia bada� wskazanych przez autor�w, okre�lone np. jako ``Dalszy rozw�j".
	\item {\bf Jakie s� sposoby ekstrakcji zmian kodu �r�d�owego na poziomie metod?}
		\\Jakie s� sposoby por�wnywania wersji kodu �r�d�owego, jakiego rodzaju dane (metryki) s� uzyskiwane.
\end{itemize}

Podczas wst�pnego rozpoznania dziedziny zauwa�ono, �e liczba publikacji jest niewielka. W~zwi�zku z~tym postanowiono przeprowadzi� wyszukiwanie w~dw�ch etapach. W~pierwszym etapie przeszukano elektroniczne zbiory, natomiast w~drugim etapie przejrzano bibliografie pozyskanych publikacji a~tak�e wszystkie publikacje ich autor�w w~celu odnalezienia dodatkowych tekst�w.

\paragraph{Przeszukiwalne zbiory cyfrowe.}

Przeszukano poni�sze zbiory z~u�yciem ustalonych wyra�e�, za pomoc� wyszukiwarek udost�pnianych w~postaci aplikacji internetowej:
\begin{itemize}
	\item IEEE Xplore,
	\item Science Direct,
	\item ACM Digital Library,
	\item Springer Link,
	\item ISI Web of Science.
	%\item Engineering Village
	%\item Wiley Online Library
\end{itemize}

Wybrano te zbiory poniewa� pokrywaj� one wi�kszo�� publikacji in�ynierii oprogramowania oraz s� u�ywane jako �r�d�a w~innych przegl�dach z~tej dziedziny \cite{hall2012systematic, riaz2009systematic}.

\paragraph{Szara literatura.}

Wst�pne przeszukiwanie wykaza�o niewielk� ilo�� �r�de� �ci�le odpowiadaj�cych zagadnieniu predykcji defekt�w na niskim poziomie granulacji. Aby pokry� znaczn� cz�� szarej literatury zdecydowano, aby przeszuka� nast�puj�ce �r�d�a:
\begin{itemize}
	\item Google Scholar.
	\item Lista odno�nik�w w~znalezionych �r�d�ach pierwotnych.
		\\Zgodnie z~metod� �nie�nej kuli \cite{goodman1961snowball} przejrzano listy referencji �r�de� pierwotnych w~celu odnalezienia dodatkowych istotnych (relewantnych) publikacji.
	\item Inne publikacje autor�w znalezionych �r�de� pierwotnych.
		\\Przeszukano baz� DBLP \cite{DBLP:2014:Online} szukaj�c wed�ug nazwisk autor�w dotychczas zgromadzonych �r�de� pierwotnych.
\end{itemize}


Po dokonaniu przegl�du literatury oraz oceny znalezionych �r�de�, wybrano te najbardziej istotne z~punktu widzenia niniejszej pracy:
\begin{itemize}
	\item \textit{Declarative visitors to ease fine-grained source code mining with full history on billions of AST nodes.} \cite{dyer2013declarative}
	\item \textit{Method-level bug prediction.} \cite{giger2012method}
	\item \textit{Comparing fine-grained source code changes and code churn for bug prediction.} \cite{giger2011comparing}
	\item \textit{Fault-prone Module Prediction Using Version Histories.} \cite{hata2012fault}
	\item \textit{Reconstructing fine-grained versioning repositories with git for method-level bug prediction.} \cite{hata2010reconstructing}
	\item \textit{Historage: fine-grained version control system for Java.} \cite{hata2011historage}
	\item \textit{Bug prediction based on fine-grained module histories.} \cite{hata2012bug}
\end{itemize}




%%%%%%%%%%%%%%%%%%%%%%%%%%%%%%%%%%%%%%%%%%%%%%%%%%%%%%%%%%%%
%%%%%%%%%%%%%%%%%%%%%%%%%%%%%%%%%%%%%%%%%%%%%%%%%%%%%%%%%%%%
\section{Eksploracja danych}

\noindent Obecnie na �wiecie gwomadzi si� ogrmone ilo�ci cyfrowych danych. Dane s� zbierane na ka�dym kroku. Wed�ug badania IDC Digital Universe \cite{gantz2012digital} w~2012 roku cyfrowy wszech�wiat osi�gn�� rozmiar 2,8 zettabajt�w (1 ZB = $10^{21}$ B), a~w~latach 2012 do 2020 roku rozmiary cyfrowego wszech�wiata b�d� si� podwaja� co dwa lata. Dane s� zbierane na ka�dym kroku: bank zapisuje wszystkie nasze operacje finansowe --- wp�aty, wyp�aty, przelewy, histori� kredytu, p�atno�ci kart�, itd.; podczas przegl�dania internetu narz�dzia analityczne zapisuj� ka�dy nasz krok; firma handlowa zapisuje w~systemie CRM (ang. \textit{Customer relationship management}) interakcje z~klientami, a~w~systemie finansowo-ksi�gowym informacje o~sprzeda�y, zakupach, produktach w~magazynie, itd.; dostawca internetu (ISP) zapisuje wszystkie nasze ��dania w~logach.

Ogromne ilo�ci danych powoduj� niemo�liwo�� ich analizy przez ludzki rozum oraz odseparowanie u�ytecznych danych od bezwarto�ciowych. Zgodnie z~przywo�anym raportem w~2012 roku mo�na by�o wykorzysta� 23\% wszystkich danych, pod warunkiem, �e by�yby one otagowane i~przeanalizowane. Jednak tylko 3\% potencjalnie u�ytecznych danych by�o otagowane a~0,5\% analizowane. Z~pomoc� przychodzi dziedzina zwana eksploracj� danych. Opiera si� ona na wykorzystaniu szybko�ci komputera do znajdowania niewidocznych dla cz�owieka prawid�owo�ci w~zgromadzonych danych. Przyk�adowe obszary zastosowania eksploracji danych:
\begin{itemize}
	\item meteorologia --- prognozowanie pogody,
	\item ekonomia --- rozpoznawanie trend�w na rynkach finansowych,
	\item medycyna --- stawianie diagnozy na podstawie symptom�w,
	\item marketing --- tworzenie reklam dopasowanych do odbiorcy,
	\item bankowo�� --- ocena ryzyka kredytowego,
	\item biotechnologia --- analiza danych genetycznych.
\end{itemize}
Powy�sza lista to bardzo ma�y wycinek mo�liwo�ci zastosowania eksploracji danych w~dzisiejszym �wiecie.

Klasyfikacja metod eksploracji danych ze wzgl�du na cel eksploracji \cite{63810}:
\begin{itemize}
	\item Odkrywanie asocjacji --- odkrywanie interesuj�cych zale�no�ci lub korelacji.
	\item Klasyfikacja i~predykcja --- odkrywanie modeli opisuj�cych zale�no�ci pomi�dzy klasyfikacj� obiekt�w a~ich charakterystyk�, w~celu ich wykorzystania do klasyfikacji nowych obiekt�w.
	\item Grupowanie --- znajdowanie zbior�w obiekt�w maj�cych podobne cechy.
	\item Analiza sekwencji i~przebieg�w czasowych --- znajdowanie cz�stych podsekwencji, trend�w, podobie�stw, anomalii oraz cykli.
	\item Odkrywanie charakterystyk --- znajdowanie zwi�z�ych opis�w og�lnych w�asno�ci klas obiekt�w.
	\item Eksploracja tekstu i~danych semistrukturalnych --- analiza danych tekstowych w~celu ich grupowania, klasyfikacji, wsparcia przeszukiwania.
	\item Eksploracja www --- znajdowanie wzorc�w zachowa� u�ytkownik�w Internetu.
	\item Eksploracja graf�w i~sieci spo�eczno�ciowych --- analiza struktur grafowych, kt�re s� szeroko wykorzystywane do modelowania z�o�onych obiekt�w, takich jak: zwi�zki chemiczne, struktury bia�kowe, sieci spo�eczno�ciowe, sieci biologiczne, itd.
	\item Eksploracja danych multimedialnych i~danych przestrzennych --- analiza i~eksploracja danych obejmuj�cych obrazy, mapy, d�wi�ki, filmy, itp.
	\item Wykrywanie punkt�w osobliwych --- wykrywanie obiekt�w, kt�re odbiegaj� od og�lnego modelu.
\end{itemize}

W niniejszej pracy wykorzystano metody nale��ce do grupy klasyfikacji i~predykcji. Klasyfikacja polega na przypisaniu zadanych element�w do ustalonych klas. Ka�dy element mo�e by� przypisany tylko do jednej klasy. W~zadaniu predykcji defekt�w oprogramowania mo�na wyr�ni� dwie klasy: ``zawiera b��d" (oznaczana dalej jako \textit{1}, \textit{pozytywna}, \textit{true}) i~``nie zawiera b��du" (oznaczana dalej jako \textit{0}, \textit{negatywna}, \textit{false}). Jest to szczeg�lny przypadek klasyfikacji, nazywany klasyfikacj� binarn�. Wynik klasyfikacji mo�na przedstawi� w~postaci macierzy pomy�ek.

\begin{table}[htbp]
	\caption{Przyk�ad macierzy pomy�ek}
	\label{macierz-pomylek}
	\begin{center}
		\begin{tabular}{|c|c|c|c|}
			\cline{3-4}
			\multicolumn{2}{c}{} & \multicolumn{2}{|c|}{przewidywany} \\
			\cline{3-4}
			\multicolumn{2}{c|}{} & 1 & 0 \\
			\cline{1-4}
			\multirow{2}{*}{rzeczywisty} & 1 & $TP$ & $FN$ \\
			\cline{2-4}
			& 0 & $FP$ & $TN$ \\
			\cline{1-4}
		\end{tabular}
	\end{center}
\end{table}

Tabela \ref{macierz-pomylek} jest przyk�adow� macierz� pomy�ek, w~kt�rej warto�ci liczbowe w~kom�rkach zosta�y zast�pione etykietami oznaczaj�cymi nast�puj�co:
\begin{itemize}
	\item TP (\textit{True Positive}) --- element poprawnie sklasyfikowany jako pozytywny;
	\item TN (\textit{True Negative}) --- element poprawnie sklasyfikowany jako negatywny;
	\item FP (\textit{False Positive}) --- element b��dnie sklasyfikowany jako pozytywny;
	\item FN (\textit{False Negative}) --- element b��dnie sklasyfikowany jako negatywny.
\end{itemize}



%%%%%%%%%%%%%%%%%%%%%%%%%%%%%%%%%%%%%%%%%%%%%%%%%%%%%%%%%%%%
\subsection{Uczenie maszynowe i~klasyfikacja}

\noindent Uczenie si� maszyn jest dziedzin� nauki z~obszaru sztucznej inteligencji. Polega ono na zastosowaniu algorytmu, kt�ry na podstawie danych wej�ciowych ma za zadanie dostarcza� wiedz� i~wnioski, a~tak�e doskonali� swoje dzia�anie. Dane wej�ciowe s� zmiennymi niezale�nymi i~s� one cechami badanych element�w. Informacja wyj�ciowa jest zmienn� zale�n� i~jest ni� wynik klasyfikacji. Algorytm dokonuj�cy klasyfikacji nazywa si� modelem predykcji, a~sam proces --- predykcj�. W~uczeniu maszynowym wyr�nia si� dwa zbiory danych:
\begin{itemize}
	\item Zbi�r treningowy --- sk�ada si� z~danych wej�ciowych (zmiennych niezale�nych) i~danych wyj�ciowych (poprawnie przypisanych klas). Ten zbi�r s�u�y do trenowania modelu predykcji czyli do uczenia.
	\item Zbi�r testowy --- sk�ada si� z~tych samych element�w co zbi�r treningowy, natomiast ma inne zastosowanie. S�u�y do ewaluacji efektywno�ci modelu predykcji. Zmienna zale�na jest por�wnywana z~wynikiem klasyfikacji, dzi�ki czemu mo�liwe jest obliczenie skuteczno�ci modelu predykcji.
\end{itemize}

\begin{figure}[htbp]
	\centering
	\includegraphics{diagrams/uczenie-predykcja.pdf}
	\caption{Wykorzystanie danych w~modelu predykcji}
\end{figure}


%%%%%%%%%%%%%%%%%%%%%%%%%%%%%%%%%%%%%%%%%%%%%%%%%%%%%%%%%%%%
%%%%%%%%%%%%%%%%%%%%%%%%%%%%%%%%%%%%%%%%%%%%%%%%%%%%%%%%%%%%
\section{Rola metryk w~in�ynierii oprogramowania}
\label{rola-metryk}

\noindent Norma IEEE 1061-1998 \cite{749159} definiuje metryk� jako ``funkcj� odwzorowuj�c� jednostk� oprogramowania w~warto�� liczbow�. Ta wyliczona warto�� jest interpretowalna jako stopie� spe�nienia pewnej w�asno�ci jako�ci jednostki oprogramowania."

W in�ynierii oprogramowania metryki s� wykorzystywane we wszystkich fazach procesu wytwarzania oprogramowania. Pozwalaj� na por�wnywanie ze sob� r�nych element�w lub r�nych projekt�w poniewa� s� danymi liczbowymi. W~fazie projektowania mog� s�u�y� m.in. do szacowania nak�adu pracy potrzebnego do realizacji projektu. W~fazie produkcji i~test�w do mierzenia jako�ci aplikacji, wydajno�ci pracy czy z�o�ono�ci programu.

Metryki mo�na podzieli� wed�ug r�nych kryeri�w. Ze wzgl�du na typ artefaktu jaki opisuj� dzieli si� je na:
\begin{itemize}
	\item Metryki produktu (inaczej metryki kodu �r�d�owego). S� bezpo�rednio wyliczane z~kodu �r�d�owego programu. Przyk�adem takich metryk s�:
	\begin{itemize}
		\item Zestaw metryk CK \cite{chidamber1994metrics}, do kt�rego nale��:
		\begin{itemize}
			\item u�rednione metody na klas� (ang. \textit{Weighted Methods per Class}, WMC),
			\item g��boko�� drzewa dziedziczenia (ang. \textit{Depth of Inheritance Tree}, DIT),
			\item liczba dzieci (ang. \textit{Number of Children}, NOC),
			\item zale�no�� mi�dzy obiektami (ang. \textit{Coupling Between Objects}, CBO),
			\item odpowiedzialno�� danej klasy (ang. \textit{Response For a~Class}, RFC),
			\item brak sp�jno�ci metod (ang. \textit{Lack of Cohesion of Methods}, LCOM).
		\end{itemize}
		\item OO --- metryki obiektowe, np.:
		\begin{itemize}
			\item liczba atrybut�w (ang. \textit{Number of attributes}, NOA),
			\item liczba metod (ang. \textit{Number of methods}, NOM),
			\item liczba dziedziczonych metod (ang. \textit{Number of methods inherited}, NOMI).
		\end{itemize}
		\item LOC --- liczba linii kodu.
	\end{itemize}
	\item Metryki procesu (inaczej metryki zmian). Okre�laj� zmienno�� atrybutu w~czasie. Oblicza si� je dla zadanych przedzia��w czasowych. Niezb�dna do ich obliczenia jest historia projektu, kt�r� mo�na uzyska� dzi�ki systemom kontroli wersji (jak SVN czy Git). Przyk�ady metryk procesu:
		\begin{itemize}
			\item liczba modyfikacji (rewizji) pliku (ang. \textit{Number of Revisions}, NR),
			\item liczba autor�w zmieniaj�cych plik (ang. \textit{Number of Distinct Commiters}, NDC),
			\item liczba zmienionych linii kodu (ang. \textit{Number of Modified Lines}, NML),
			\item wiek pliku (ang. \textit{Age}, AGE),
			\item liczba refaktoryzacji pliku (ang. \textit{(Number of Refactorings}, NREF),
			\item liczba dodanych (usuni�tych, zmienionych) metod,
			\item liczba dodanych (usuni�tych, zmienionych) atrybut�w.
		\end{itemize}
\end{itemize}

Dodatkowo mo�na podzieli� metryki z~uwagi na cel pomiaru \cite{gorski2000inzynieria}:
\begin{itemize}
	\item metryki z�o�ono�ci,
	\item metryki szacowania nak�adu,
	\item metryki funkcjonalno�ci.
\end{itemize}


Model predykcji defekt�w to narz�dzie, kt�re na podstawie warto�ci metryk danego projektu dokonuje wskazania defekt�w znajduj�cych si� w~tym projekcie. Aby poprawnie zinterpretowa� wskazania dostarczane przez model predykcji defekt�w nale�y okre�li� czym jest defekt. Norma 982.2 IEEE/ANSI \cite{26479} definiuje defekt jako anomali� w~produkcie, kt�ra mo�e by�:
\begin{itemize}
	\item zaniechaniami i~niedoskona�o�ciami znalezionymi podczas wczesnych faz cyklu �ycia oraz
	\item b��dami zawartymi w~oprogramowaniu wystarczaj�co dojrza�ym do testowania lub dzia�ania.
\end{itemize}

Istniej�ce badania wykaza�y, �e metryki procesu przewy�szy�y metryki produktu w~kontek�cie budowania modeli predykcji defekt�w \cite{giger2012method, kamei2010revisiting, mende2009revisiting, ferzund2009empirical}. Z~tego powodu w~dalszej cz�ci pracy zrezygnowano z~wykorzystania metryk produktu, bior�c pod uwag� jedynie metryki procesu.






%%%%%%%%%%%%%%%%%%%%%%%%%%%%%%%%%%%%%%%%%%%%%%%%%%%%%%%%%%%%
%%%%%%%%%%%%%%%%%%%%%%%%%%%%%%%%%%%%%%%%%%%%%%%%%%%%%%%%%%%%
\section{Koszty zapewnienia jako�ci}

\noindent W~in�ynierii oprogramowania wyst�puje kilka r�nych definicji jako�ci. Na potrzeby niniejszej pracy przyj�to definicj� Kana \cite{kan2002metrics} ``brak defekt�w w~produkcie". Zarz�dzanie jako�ci� oprogramowania polega na podejmowaniu dzia�a� maj�cych na celu zapewnienie jako�ci tworzonego oprogramowania poprzez szereg test�w, kt�re wspieraj� ca�y proces rozwoju oprogramowania.
\begin{itemize}
	\item Na etapie zbierania wymaga� --- weryfikacja czy okre�lone wymagania b�d� mo�liwe do zweryfikowania (przetestowania).
	\item Na etapie projektowania --- zaplanowanie procesu testowego, wyb�r �rodowisk testowych.
	\item Na etapie kodowania --- definiowanie i~realizacja scenariuszy i~przypadk�w testowych oraz rejestracja defekt�w.
	\item Na etapie zamkni�cia projektu --- testy integracyjne, testy akceptacyjne, testy operacyjne.
\end{itemize}

\noindent Jak wykazano w~\cite{arisholm2010systematic} koszty zapewnienia jako�ci s� prawie proporcjonalne do wielko�ci modu�u. Dlatego badacze bior� pod uwag� wysi�ek zwi�zany z~dzia�aniami maj�cymi na celu zapewnienie jako�ci \cite{rahman2011bugcache, koru2008theory, menzies2010defect}. Zmniejszenie wysi�ku i~kosztu zwi�zanego z~zapewnieniem jako�ci to obecnie jeden z~g��wnych kierunk�w bada� \cite{hata2012bug}.

Podstawowym celem pracy dyplomowej jest stworzenie narz�dzi w~postaci wtyczek do �rodowiska KNIME, s�u��cych do gromadzenia metryk oprogramowania z~system�w kontroli wersji oraz zgromadzenie jak najwi�kszej ilo�ci metryk w~publicznym repozytorium. Nast�pnym krokiem jest stworzenie modelu (modeli) predykcji defekt�w oraz ich ewaluacja, bior�c pod uwag� wysi�ek zwi�zany z~zapewnieniem jako�ci oprogramowania. Stworzenie narz�dzi pozwalaj�cych na zautomatyzowane gromadzenie metryk z~dost�pnych projekt�w (na przyk�ad open source) pozwoli rozszerzy� publiczne zbiory danych. Dzi�ki temu b�dzie mo�liwe wykorzystanie tych danych do tworzenia modeli predykcji defekt�w oprogramowania dzi�ki: wi�kszym zbiorom ucz�cym; ewaluacji modeli na wi�kszych zbiorach danych.







%%%%%%%%%%%%%%%%%%%%%%%%%%%%%%%%%%%%%%%%%%%%%%%%%%%%%%%%%%%%
%%%%%%%%%%%%%%%%%%%%%%%%%%%%%%%%%%%%%%%%%%%%%%%%%%%%%%%%%%%%
\section{Systemy kontroli wersji jako �r�d�o danych o~projektach}

\noindent Jak wspomniano wcze�niej w~rozdziale \ref{rola-metryk} aby obliczy� metryki procesu, konieczne jest uzyskanie historii projektu. Przegl�d literatury pozwoli� na wyodr�bnienie sposob�w i~narz�dzi, kt�re pozwalaj� na por�wnywanie r�nych wersji kodu �r�d�owego. Jak wykazano w~\cite{kamei2010revisiting, posnett2011ecological, nguyen2010studying} predykcja na poziomie plik�w jest bardziej efektywna ni� na poziomie pakiet�w. Id�c dalej w~kierunku uszczeg�owienia wynik�w predykcji, mo�na przypuszcza�, �e predykcja na poziomie metod by�aby skuteczniejsza ni� na poziomie plik�w. Badanie \cite{hata2012bug} wykaza�o, �e pliki zawieraj�ce b��dy zawieraj� prawie lub ponad 10 metod, natomiast tylko kilka metod zawiera b��dy (mediana 1--2). Jest to nie tylko odpowied� na pytanie czy predykcja na poziomie metod jest skuteczniejsza, ale r�wnie� wskazanie przyczyny takiego stanu rzeczy. Jednak�e aby w~pe�ni wykorzysta� mo�liwo�ci ograniczenia koszt�w jako�ci poprzez predykcj� na poziomie metod, potrzebne s� skuteczne modele, dostarczaj�ce wiarygodnych wynik�w.

Ze wzgl�du na powy�sze zale�no�ci, podj�to decyzj� o~prowadzeniu dalszych prac w~kierunku budowy modeli predykcji na poziomie metod. Poni�ej wypisano techniki por�wnywania kodu �r�d�owego na poziomie metod.
\begin{itemize}
	\item ChangeDistiller \cite{fluri2007change} --- polega na odwzorowaniu kodu �r�d�owego Java w~strukturze drzewiastej, jak� jest AST (ang. \textit{Abstract Syntax Tree}) a~nast�pnie wyodr�bnieniu zmian pomi�dzy dwiema wersjami przy u�yciu algorytm�w por�wnywania drzew.
	\item Historage \cite{hata2011historage} --- wykorzystuje system kontroli wersji Git do przechowywania zidentyfikowanych zmian w~kodzie na niskim poziomie.
	\item APFEL \cite{zimmermann2006fine} --- jest wtyczk� do �rodowiska Eclipse, kt�ra zbiera w~bazie danych niskopoziomowe zmiany w~kodzie. Dzia�a z~systemem kontroli wersji CVS i~�r�d�ami Java.
	\item C-REX \cite{hassan2004c} --- wyodr�bnia fakty z~historii kodu �r�d�owego j�zyka C, a~nast�pnie por�wnuje ze sob� kolejne wersje.
	\item Kenyon \cite{bevan2005facilitating}.
	\item Beagle \cite{godfrey2005using}.
\end{itemize}






%%%%%%%%%%%%%%%%%%%%%%%%%%%%%%%%%%%%%%%%%%%%%%%%%%%%%%%%%%%%
\subsection{Historia b��d�w w~projektach}
\label{linkowanie-bledow}

\noindent Metryki, kt�re stanowi� dane wej�ciowe w~modelach predykcji s� zmiennymi niezale�nymi (ang. \textit{independent variables}). Pe�ny zestaw danych potrzebny do wytrenowania modelu obejmuje tak�e zmienne zale�ne (ang. \textit{dependent variables}). Zmienna niezale�na reprezentuje wyj�cie (wynik), oraz mo�e by� u�ywana do testowania modelu, �eby oceni� jego skuteczno��. W~predykcji defekt�w oprogramowania zmienn� zale�n� jest liczba b��d�w lub zmienna okre�laj�ca czy wyst�puje b��d.

Aby uzyska� informacje o~b��dach w~projekcie stosuje si� metody linkowania b��d�w. Linkowanie polega na odszukaniu powi�za� pomi�dzy zmian� zapisan� w~repozytorium kodu, a~b��dem zg�oszonym w~systemie �ledzenia zmian (ang. \textit{Issue Tracking System}, ITS), takim jak JIRA, Bugzilla, IBM Rational ClearQuest czy innym.

Metoda u�ywana w~tej pracy opiera si� na metodzie SZZ \cite{sliwerski2005changes}. Jej podstawow� zalet� jest por�wnywanie czasu naprawienia b��du zapisanego w~ITS z~czasem wys�ania poprawki do systemu kontroli wersji. Dzi�ki takiemu por�wnaniu wyklucza si� du�� liczb� b��dnych wskaza�, kt�re wynikaj� z~niew�a�ciwego lub przypadkowego przyporz�dkowania b��du do zmiany kodu. Przyczyn� takich b��dnych dopasowa� mo�e by� umieszczenie w~opisie zmiany ci�gu numerycznego nieb�d�cego numerem b��du.






%%%%%%%%%%%%%%%%%%%%%%%%%%%%%%%%%%%%%%%%%%%%%%%%%%%%%%%%%%%%
\chapter{Om�wienie infrastruktury pomiarowej}
\label{rozdzial3}
\noindent
Badania zgodnie z~za�o�eniami przeprowadzono w~�rodowisku R. Do zebrania danych z~projekt�w u�yto �rodowiska KNIME rozszerzonego o~wtyczki z~DePress.

%%%%%%%%%%%%%%%%%%%%%%%%%%%%%%%%%%%%%%%%%%%%%%%%%%%%%%%%%%%%
%%%%%%%%%%%%%%%%%%%%%%%%%%%%%%%%%%%%%%%%%%%%%%%%%%%%%%%%%%%%
\section{KNIME}

\noindent KNIME jest platform� do analizy danych, kt�ra umo�liwia wykonywanie zaawansowanych statystyk i~eksploracji danych, w~celu analizy trend�w i~przewidywania wynik�w. Wizualny zestaw narz�dzi pozwala na pozyskiwani danych, przekszta�canie ich, wst�pne rozpoznanie, analizy predykcyjne i~wizualizacj�. KNIME daje r�wnie� mo�liwo�� tworzenia raport�w na podstawie zgromadzonych informacji. KNIME Desktop jest oprogramowaniem o~otwartym kodzie (ang. \textit{open source}) udost�pnianym na licencji GPL \cite{BCDG+07}.

\begin{figure}[htbp]
	\centering
	\includegraphics[width=1\textwidth]{img/knime-gui}
	\caption{Interejs KNIME. �r�d�o \cite{BCDG+07}}
\end{figure}

Platforma KNIME zawiera setki w�z��w pozyskiwania danych, przetwarzania i~filtrowania, analizy i~eksploracji danych, wizualizacji i~innych. Oprogramowanie bazuje na platformie Eclipse\footnote{http://eclipse.org} i~jest rozszerzalne poprzez system wtyczek, dzi�ki czemu znajduje zastosowanie w~komercyjnych �rodowiskach produkcyjnych jak i~�rodowiskach badawczych. Przyk�ad procesu analizy i~eksploracji danych przedstawiono na rysunku \ref{knime-workflow}.

\begin{figure}[htbp]
	\centering
	\includegraphics[width=1\textwidth]{img/knime-workflow}
	\caption{Proces zbudowany w~KNIME. �r�d�o \cite{BCDG+07}}
	\label{knime-workflow}
\end{figure}

\paragraph{Budowanie procesu.} Proces jest tworzony z~w�z��w dost�pnych w~programie. W�z�y umieszcza si� w~edytorze i~��czy ze sob�. S� one podstawowymi jednostkami w~procesie. Ka�dy w�ze� mo�e mie� porty wej�ciowe i~wyj�ciowe. Dane s� przekazywane pomi�dzy w�z�ami z~portu wyj�ciowego jednego w�z�a do portu wej�ciowego innego w�z�a. Wyj�cie w�z�a mog� stanowi� na przyk�ad dane tabelaryczne, obraz, model zapisany w~PMML (ang. \textit{Predictive Model Markup Language}) i~inne.

\begin{figure}[htbp]
	\centering
	\includegraphics[width=1\textwidth]{img/knime-connected-nodes}
	\caption{W�z�y KNIME. �r�d�o \cite{BCDG+07}}
\end{figure}

W�ze� przed u�yciem musi zosta� skonfigurowany. Ka�dy rodzaj w�z�a posiada sw�j zestaw parametr�w konfigurowalnych za pomoc� interfejsu graficznego. Prawid�owo skonfigurowany w�ze� mo�e utowrzy� na wyj�ciu tabel� danych, k�tra mo�e by� ogl�dana w~wewn�trznym edytorze. Opr�cz tabeli danych niekt�re w�z�y udost�pniaj� widoki. Widokiem mo�e by� np. wykres, graf, tabela, itd. Widoki mog� zawiera� elementy interaktywne pozwalaj�ce na dostosowywanie prezentowanych danych, zmian� uk�adu, kolor�w, typu wykresu, itp.



%%%%%%%%%%%%%%%%%%%%%%%%%%%%%%%%%%%%%%%%%%%%%%%%%%%%%%%%%%%%
%%%%%%%%%%%%%%%%%%%%%%%%%%%%%%%%%%%%%%%%%%%%%%%%%%%%%%%%%%%%
\section{R}

\noindent R~jest darmowym �rodowiskiem i~j�zykiem do oblicze� statystycznych i~wizualizacji. Jest podobny do j�zyka i~�rodowiska S. Ma szerokie zastosowanie na �wiecie, jest podstawowym narz�dziem w~bioinformatyce, u�ywa si� go w~takich firmach jak Facebook, Google, Form, Mozilla czy Twitter. Wiele pakiet�w statystycznych (np. RapidMiner, KNIME) oferuje mechanizmy zapewniaj�ce wsp�prac� z~R.

R jest zbudowany w~architekturze modu�owej. Jest rozszerzalny za pomoc� pakiet�w zawieraj�cych dodatkowe funkcje lub dodatkowe narz�dzia przeznaczone dla poszczeg�lnych dziedzin nauki. Zalet� R~jest tak�e mo�liwo�� tworzenia wysokiej jako�ci grafik gotowych do publikacji. Tworzenie grafik jest uproszczone poprzez stosowanie gotowych szablon�w, z~jednoczesnym zachowaniem mo�liwo�ci pe�nej kontroli i~personalizacji przez u�ytkownika. Ma tak�e sw�j w�asny format dokumentacji podobny do LaTeX.




%%%%%%%%%%%%%%%%%%%%%%%%%%%%%%%%%%%%%%%%%%%%%%%%%%%%%%%%%%%%
%%%%%%%%%%%%%%%%%%%%%%%%%%%%%%%%%%%%%%%%%%%%%%%%%%%%%%%%%%%%
\section{DePress}
\label{depress}

\noindent DePress jest oparty wy��cznie na architekturze wtyczek KNIME, dzi�ki czemu mo�liwa jest integracja i~wsp�dzia�anie z~istniej�cymi wtyczkami. G��wnym zadaniem pakietu jest predykcja defekt�w oprogramowania. Poszczeg�lne wtyczki dostarczaj� odpowiednich funkcji w~zale�no�ci od ich przeznaczenia. Podzielono je na trzy g��wne grupy (rysunek \ref{depress-struktura}):
\begin{itemize}
	\item Adaptery --- pozwalaj� na pobieranie danych z~zewn�trznych narz�dzi.
	\item Generatory metryk --- wyliczaj� metryki na podstawie danych wej�ciowych.
	\item Inne --- dostarczaj� metod do dodatkowych przekszta�ce� danych.
\end{itemize}

\begin{figure}[htbp]
	\centering
	\includegraphics[width=1\textwidth]{diagrams/depress.pdf}
	\caption{Struktura DePress. �r�d�o \cite{madeyski2014software}}
	\label{depress-struktura}
\end{figure}

Dzi�ki separacji wtyczek jest mo�liwa �atwa zamiana jednej wtyczki na inn�. Na przyk�ad gdy analizuj�c projekt oka�e si�, �e zosta� on przeniesiony z~SVN do Git, wystarczy u�y� odpowiedniej wtyczki adaptera aby na nowo pobra� dane o~projekcie.

W badaniu u�yto wtyczek adapter�w Jira (Online) i~Bugzilla (Online) do pobierania danych o~zg�oszonych b��dach z~system�w �ledzenia zagadnie� (ang. \textit{Issue Tracking System}, ITS). Istniej�ce wtyczki adapter�w do wsp�pracy z~systemami kontroli wersji (SCM) okaza�y si� nie wystarczaj�ce, poniewa� dostarczaj� tylko danych o~zmianach na poziomie klas. Potrzebne by�o stworzenie nowej wtyczki AstMetrics do ekstrakcji zmian na niskim poziomie granulacji.


%%%%%%%%%%%%%%%%%%%%%%%%%%%%%%%%%%%%%%%%%%%%%%%%%%%%%%%%%%%%
%%%%%%%%%%%%%%%%%%%%%%%%%%%%%%%%%%%%%%%%%%%%%%%%%%%%%%%%%%%%
\section{AstMetrics}
\label{metryki}

\noindent Wtyczka AstMetrics w~pierwszej wersji zosta�a napisana przez Piotra Mitk� \cite{mitka}. Na potrzeby tego badania zosta�a w~du�ej cz�ci napisana od nowa i~poszerzona o~nowe metryki. Bazuj�c na oprogramowaniu Change Distiller \cite{fluri2007change} dokonuje ona por�wnania historycznych wersji kodu �r�d�owego ka�dego pliku pobranego z~repozytorium. Ka�da wersja pliku jest odwzorowywana za pomoc� drzewa AST (ang. \textit{Abstract Syntax Tree}) a~nast�pnie drzewa s� por�wnywane pomi�dzy kolejnymi wersjami (patrz rysunek \ref{ast-compare}). Na tej podstawie jest mo�liwe wykrycie zmian w~kodzie �r�d�owym na niskim poziomie granulacji. Zmiany s� zapisywane w~tymczasowej bazie danych a~nast�pnie s� wyliczane z~nich metryki dla ka�dej metody z~ka�dej klasy znajduj�cej si� w~kodzie �r�d�owym.

\begin{figure}[htbp]
	\centering
	\includegraphics[width=1\textwidth]{img/ast-compare}
	\caption{Por�wnanie drzewa AST. �r�d�o \cite{fluri2007change}}
	\label{ast-compare}
\end{figure}

Metryki wyliczane w~pierwszej wersji wtyczki zosta�y zawarte w~tabeli \ref{metryki1} a~dodane w~drugiej wersji wtyczki w~tabeli \ref{metryki2}.


\begin{table}[htbp]
	\caption{Metryki wyliczane w~pierwszej wersji wtyczki AstMetrics}
	\label{metryki1}
	\begin{center}
		\begin{tabular}{@{}ll@{}}
			\toprule
			\bf Nazwa metryki &
			\bf Opis \\
			\midrule
			allMethodHistories & ile razy metoda by�a zmieniana w~historii                   \\
			methodHistories    & ile razy metoda by�a zmieniana w~danych przedziale czasowym \\
			authors            & liczba autor�w                                              \\
			stmtAdded          & ��czna liczba dodanych instrukcji                           \\
			maxStmtAdded       & maksymalna liczba dodanych instrukcji                       \\
			avgStmtAdded       & �rednia liczba dodanych instrukcji                          \\
			stmtUpdated        & ��czna liczba zmienionych instrukcji                        \\
			maxsSmtUpdated     & maksymalna liczba zmienionych instrukcji                    \\
			avgStmtUpdated     & �rednia liczba zmienionych instrukcji                       \\
			stmtDeleted        & ��czna liczba usuni�tych instrukcji                         \\
			maxStmtDeleted     & maksymalna liczba usuni�tych instrukcji                     \\
			avgStmtDeleted     & �rednia liczba usuni�tych instrukcji                        \\
			stmtParentChanged  & ��czna liczba zmienionych instrukcji nadrz�dnych            \\
			churn              & suma stmtAdded i~stmtDeleted we wszystkich wersjach         \\
			maxChurn           & maksymalna warto�� churn na przestrzeni wersji              \\
			avgChurn           & �rednia warto�� churn na przestrzeni wersji                 \\
			decl               & ��czna liczba zmian deklaracji metody                       \\
			cond               & ��czna liczba zmian deklaracji warunkowych                  \\
			elseAdded          & ��czna liczba dodanych cz�ci \textit{else}                 \\
			elseDeleted        & ��czna liczba usuni�tych cz�ci \textit{else}               \\
			\bottomrule
		\end{tabular}
	\end{center}
\end{table}

\begin{table}[htbp]
	\caption{Metryki dodane w~drugiej wersji wtyczki AstMetrics}
	\label{metryki2}
	\begin{center}
		\begin{tabular}{@{}ll@{}}
			\toprule
			\bf Nazwa metryki &
			\bf Opis \\
			\midrule
			loopsAdded & ��czna liczba dodanych p�tli \\
			loopsUpdated & ��czna liczba zmienionych p�tli \\
			loopsDeleted & ��czna liczba usuni�tych p�tli \\
			variablesAdded & ��czna liczba dodanych instrukcji deklaracji zmiennej \\
			variablesUpdated & ��czna liczba zmienionych instrukcji deklaracji zmiennej \\
			variablesDeleted & ��czna liczba usuni�tych instrukcji deklaracji zmiennej \\
			assigmentsAdded & ��czna liczba dodanych instrukcji przypisania \\
			assigmentsUpdated & ��czna liczba zmienionych instrukcji przypisania \\
			assigmentsDeleted & ��czna liczba usuni�tych instrukcji przypisania \\
			returnsAdded & ��czna liczba dodanych instrukcji \textit{return} \\
			returnsUpdated & ��czna liczba zmienionych instrukcji \textit{return} \\
			returnsDeleted & ��czna liczba usuni�tych instrukcji \textit{return} \\
			nullsAdded & ��czna liczba dodanych instrukcji zawieraj�cych \textit{null} \\
			nullsUpdated & ��czna liczba zmienionych instrukcji zawieraj�cych \textit{null} \\
			nullsDeleted & ��czna liczba usuni�tych instrukcji zawieraj�cych \textit{null} \\
			casesAdded & ��czna liczba dodanych instrukcji warunkowych \textit{case} \\
			casesUpdated & ��czna liczba zmienionych instrukcji warunkowych \textit{case} \\
			casesDeleted & ��czna liczba usuni�tych instrukcji warunkowych \textit{case} \\
			breaksAdded & ��czna liczba dodanych instrukcji \textit{break} \\
			breaksUpdated & ��czna liczba zmienionych instrukcji \textit{break} \\
			breaksDeleted & ��czna liczba usuni�tych instrukcji \textit{break} \\
			objectsAdded & ��czna liczba dodanych instrukcji tworz�cych obiekt \\
			objectsUpdated & ��czna liczba zmienionych instrukcji tworz�cych obiekt \\
			objectsDeleted & ��czna liczba usuni�tych instrukcji tworz�cych obiekt \\
			catchesAdded & ��czna liczba dodanych blok�w przechwytywania wyj�tku \\
			catchesUpdated & ��czna liczba zmienionych blok�w przechwytywania wyj�tku \\
			catchesDeleted & ��czna liczba usuni�tych blok�w przechwytywania wyj�tku \\
			throwsAdded & ��czna liczba dodanych blok�w rzucania wyj�tku \\
			throwsUpdated & ��czna liczba zmienionych blok�w rzucania wyj�tku \\
			throwsDeleted & ��czna liczba usuni�tych blok�w rzucania wyj�tku \\
			\bottomrule
		\end{tabular}
	\end{center}
\end{table}

Dodatkowo zaimplementowano drugie wyj�cie wtyczki z~wszystkimi historiami metod, tj. z~informacjami o~tym kt�ra metoda zosta�a zmieniona w~danej wersji historycznej w~repozytorium kodu. W~zwi�zku z~tym, �e do repozytorium s� zapisywane zmiany w~pojedycznych plikach, a~nie bardziej szczeg�owe, wyst�pi�a potrzeba stworzenia takiego zestawienia. Struktura tworzonej tabeli danych zosta�a przedstawiona w~tabeli \ref{ast-histories}. Dane historyczne s�u�� do wyszukiwania defekt�w w~kodzie �r�d�owym, kt�re zosta�o opisane w~rozdziale \ref{wyszukiwanie-defektow}.

\begin{table}[htbp]
	\caption{Struktura tabeli wyj�ciowej z~danymi historycznymi wtyczki AstMetrics}
	\label{ast-histories}
	\begin{center}
		\begin{tabular}{@{}ll@{}}
			\toprule
			\bf Nazwa kolumny &
			\bf Opis \\
			\midrule
			MethodName & Nazwa metody                  \\
			Author     & Autor zmiany                  \\
			Message    & Opis zmiany                   \\
			Date       & Data przes�ania               \\
			CommitID   & Unikalny identyfikator zmiany \\
			\bottomrule
		\end{tabular}
	\end{center}
\end{table}

Do prawid�owego dzia�ania wtyczki wymagane jest okre�lenie warto�ci parametr�w:
\begin{itemize}
	\item dirname - katalog ze sklonowanym repozytorium git (np. \textit{C:/ant} lub \textit{/home/user/ant},
	\item package - pakiet, kt�ry zostanie uwzgl�dniony w~tabelach wyj�ciowych (np. \textit{org.apache.ant}),
	\item exclude\_dir - katalog, kt�ry zostanie wy��czony z~pobieranai danych (np. \textit{src/tests}),
	\item bottom\_commit - identyfikator rewizji pocz�tkowej, dozwolone jest podanie identyfikatora SHA-1, tagu poprzedzonego napisem \textit{tag:} lub napis \textit{initial} oznaczaj�cego pierwsz� (najstarsz�) rewizj� w~repozytorium,
	\item top\_commit - identyfikator rewizji ko�cowej, czyli rewizji dla kt�rej zostan� obliczone metryki, dozwolone jest podanie identyfikatora SHA-1 lub tagu poprzedzonego napisem \textit{tag:},
	\item top\_commit\_post\_rel - identyfikator rewizji zamykaj�cej, pomi�dzy t� wersj� a~wersj� \textit{top\_commit} zostanie pobrana historia repozytorium, dozwolone jest podanie identyfikatora SHA-1, tagu poprzedzonego napisem \textit{tag:} lub napis \textit{current} oznaczaj�cego ostatni� (najnowsz�) rewizj� w~repozytorium.
\end{itemize}

Dzia�anie wtyczki obrazuje rysunek \ref{ast-metrics-process}.


\begin{figure}[htbp]
	\centering
	\includegraphics[width=1\textwidth]{diagrams/ast-metrics-process.pdf}
	\caption{Diagram aktywno�ci wtyczki AstMetrics}
	\label{ast-metrics-process}
\end{figure}



%%%%%%%%%%%%%%%%%%%%%%%%%%%%%%%%%%%%%%%%%%%%%%%%%%%%%%%%%%%%
\subsection{Gromadzenie metryk}

\noindent Do trenowania i~oceny algorytm�w klasyfikacji cz�sto stosuje si� walidacj� krzy�ow�, polegaj�c� na podziale zbioru danych na dwa podzbiory --- treningowy i~testowy --- a~nast�pnie wykorzystaniu ich zgodnie z~przeznaczeniem. W~niniejszej pracy zastosowano inne podej�cie, polegaj�ce na wykorzystaniu dw�ch niezale�nych zbior�w danych z~dw�ch r�nych wyda� oprogramowania. Metryki wersji A~pos�u�y�y jako zbi�r treningowy a~metryki wersji B~jako zbi�r testowy. Zalet� takiego podej�cia jest lepsze dopasowanie do rzeczywistych warunk�w wyst�puj�cych w~procesie wytwarzania oprogramowania. W~tabeli \ref{wersje} zawarto informacje o~przebadanym oprogramowaniu wraz z~oznaczeniami wersji A~i~B. Metryki zosta�y wyliczone z~u�yciem wtyczki AstMetrics. Wprowadzono r�wnie� dodatkowo� metryk� pomocnicz� LOC (ang. \textit{ang. Lines of code}) --- liczba linii kodu.

\begin{table}[htbp]
	\small
	\caption{Przegl�d projekt�w wykorzystanych do badania}
	\label{wersje}
	\begin{center}
		\begin{tabular}{@{}llll@{}}
			\toprule
			\bf Nazwa projektu &
			\bf Wersja A &
			\bf Wersja B &
			\bf ITS
			\\
			\midrule
			ECF                  & Root\_Release\_3\_0        & Root\_Release\_3\_1        & Bugzilla \\
			Ant                  & ANT\_170                   & ANT\_180                   & Bugzilla \\
			JMeter               & v2\_8                      & v2\_9                      & Bugzilla \\
			Commons Lang         & LANG\_3\_0                 & LANG\_3\_1                 & JIRA     \\
			AspectJ              & V1\_5\_0\_final            & V1\_6\_0                   & Bugzilla \\
			CXF                  & cxf-2.7.0                  & cxf-3.0.0                  & JIRA     \\
			Isis                 & isis-1.7.0                 & isis-1.8.0                 & JIRA     \\
			Commons Email        & EMAIL\_1\_2                & EMAIL\_1\_3                & JIRA     \\
			Commons JXPath       & JXPATH\_1\_1               & JXPATH\_1\_2               & JIRA     \\
			Commons Logging      & LOGGING\_1\_0              & LOGGING\_1\_1\_0           & JIRA     \\
			Commons Net          & NET\_2\_0                  & NET\_2\_2                  & JIRA     \\
			Accumulo             & 1.5.2                      & 1.6.0                      & JIRA     \\
			Karaf                & karaf-2.3.0                & karaf-2.4.0                & JIRA     \\
			ActiveMQ             & activemq-5.8.0             & activemq-5.9.0             & JIRA     \\
			Hive                 & release-1.0.0              & release-1.1.0              & JIRA     \\
			James Mailbox        & apache-james-mailbox-0.3   & apache-james-mailbox-0.4   & JIRA     \\
			Jackrabbit FileVault & jackrabbit-filevault-3.0.0 & jackrabbit-filevault-3.1.0 & JIRA     \\
			OpenJPA              & 1.0.0                      & 2.0.0                      & JIRA     \\
			Jackrabbit Oak       & jackrabbit-oak-1.0.0       & jackrabbit-oak-1.1.0       & JIRA     \\
			TomEE                & tomee-1.5.0                & tomee-1.6.0                & JIRA     \\
			Tika                 & 1.5                        & 1.6                        & JIRA     \\
			Lucene - Core        & lucene\_solr\_4\_9\_0      & lucene\_solr\_4\_10\_0     & JIRA     \\
			Solr                 & lucene\_solr\_4\_9\_0      & lucene\_solr\_4\_10\_0     & JIRA     \\
			Mahout               & mahout-0.6                 & mahout-0.7                 & JIRA     \\
			Apache Gora          & apache-gora-0.3            & apache-gora-0.4            & JIRA     \\
			Flume                & release-1.4.0              & release-1.5.0              & JIRA     \\
			Nutch                & release-2.1                & release-2.2                & JIRA     \\
			Apache Knox          & v0.4.0-release             & v05.0-release              & JIRA     \\
			Phoenix              & v4.2.0                     & v4.3.0                     & JIRA     \\
			\bottomrule
		\end{tabular}
	\end{center}
\end{table}



%%%%%%%%%%%%%%%%%%%%%%%%%%%%%%%%%%%%%%%%%%%%%%%%%%%%%%%%%%%%
\subsection{Gromadzenie danych o~defektach}
\label{wyszukiwanie-defektow}

\noindent Pe�na informacja wej�ciowa do trenowania i~testowania algorytmu klasyfikacji musi zawiera� jeszcze dane o~b��dach. Aby uzyska� te dane zastosowano metod� linkowania b��d�w wed�ug poni�szego algorytmu:
\begin{enumerate}
	\item Wylistowanie wszystkich metod wchodz�cych w~sk�ad wydanej wersji oprogramowania.
	\item Pobranie historii metod za pomoc� AstMetrics od daty wydania wersji do daty ko�cowej projektu.
	\item Znalezienie rewizji naprawiaj�cych b��d.
	\item Por�wnanie daty naprawienia b��du z~dat� ustawion� w~systemie �ledzenia zagadnie� i~wyeliminowanie niezgodnych wpis�w. Niezgodny wpis to taki, kt�ry w~systemie �ledzenia zagadnie� ma przypisan� inn� dat� naprawienia b��du ni� data rewizji w~repozytorium. Za�o�ono tolerancj� wynosz�c� 7~dni, poniewa� b��d mog� zosta� oznaczony jako rozwi�zany z~pewnym op�nieniem.
	\item Zliczenie unikalnych numer�w b��du dla danej metody i~zapisanie warto�ci jako liczba b��d�w.
	\item Oznaczenie metod, kt�re zosta�y zmienione w~commitach naprawiaj�cych b��d jako metod zawieraj�cych b��d (\textit{1}).
	\item Oznaczenie pozosta�ych metod jako nie zawieraj�cych b��du (\textit{0}).
\end{enumerate}

Proces zosta� zaprojektowany w~�rodowisku KNIME, w~dw�ch wersjach: z~systemem �ledzenia zagadnie� Bugzilla i~JIRA. Proces by� uruchamiany automatycznie z~poziomu �rodowiska R, dzi�ki wykorzystaniu trybu wsadowego KNIME (ang. \textit{batch mode}). Parametry konfiguracyjne by�y przekazywane za pomoc� zmiennych (ang. \textit{workflow variables}). Wsp�dzia�anie �rodowiska KNIME i~R zosta�o zobrazowane na rysunku \ref{r-knime}.

\begin{figure}[htbp]
	\centering
	\includegraphics[width=0.5\textwidth]{diagrams/r-knime.pdf}
	\caption{Przebieg badania w �rodowisku R i KNIME}
	\label{r-knime}
\end{figure}

W tabeli \ref{wielkosc-projektow} zebrano dane dotycz�ce wielko�ci projekt�w oraz liczby b��d�w. Dane dotycz� wersji B~ka�dego projektu.

\begin{table}[htbp]
	\small
	\caption{Statystyki projekt�w}
	\label{wielkosc-projektow}
	\begin{center}
		\begin{tabular}{@{}lrrrrr@{}}
			\toprule
			\multirow{2}{*}{\bf Nazwa projektu} &
			\multirow{2}{*}{\bf LOC} &
			\bf Liczba &
			\bf Liczba &
			\bf B��dnych &
			\bf Udzia�
			\\
			&
			&
			\bf metod &
			\bf b��d�w &
			\bf metod &
			\bf b��dnych metod
			\\
			\midrule
			ECF                  & 54 264  & 9 306  & 290   & 236 & 2,54\% \\
			Ant                  & 60 715  & 9 120  & 480   & 407 & 4,46\% \\
			JMeter               & 50 181  & 7 592  & 447   & 365 & 4,81\% \\
			Commons Lang         & 15 802  & 2 555  & 37    & 34  & 1,33\% \\
			AspectJ              & 120 367 & 18 090 & 595   & 435 & 2,40\% \\
			CXF                  & 313 115 & 40 141 & 1 225 & 759 & 1,89\% \\
			Isis                 & 111 699 & 25 449 & 19    & 18  & 0,07\% \\
			Commons Email        & 1 202   & 210    & 13    & 9   & 4,29\% \\
			Commons JXPath       & 12 652  & 1 531  & 35    & 35  & 2,29\% \\
			Commons Logging      & 1 638   & 243    & 6     & 6   & 2,47\% \\
			Commons Net          & 5 536   & 1 169  & 94    & 78  & 6,67\% \\
			Accumulo             & 38 483  & 3 816  & 231   & 153 & 4,01\% \\
			Karaf                & 192 685 & 28 565 & 661   & 544 & 1,90\% \\
			ActiveMQ             & 66 861  & 10 582 & 116   & 102 & 0,96\% \\
			Hive                 & 104 202 & 15 254 & 381   & 296 & 1,94\% \\
			James Mailbox        & 14 258  & 2 130  & 20    & 20  & 0,94\% \\
			Jackrabbit FileVault & 25 861  & 3 344  & 55    & 34  & 1,02\% \\
			OpenJPA              & 174 748 & 36 954 & 1 378 & 944 & 2,55\% \\
			Jackrabbit Oak       & 116 458 & 14 971 & 983   & 731 & 4,88\% \\
			TomEE                & 9 535   & 1 091  & 63    & 56  & 5,13\% \\
			Tika                 & 31 166  & 3 797  & 226   & 212 & 5,58\% \\
			Lucene - Core        & 238 454 & 27 438 & 650   & 493 & 1,80\% \\
			Solr                 & 145 004 & 14 889 & 161   & 131 & 0,88\% \\
			Mahout               & 60 917  & 7 812  & 370   & 336 & 4,30\% \\
			Apache Gora          & 11 621  & 2 236  & 11    & 11  & 0,49\% \\
			Flume                & 40 956  & 4 791  & 115   & 107 & 2,23\% \\
			Nutch                & 15 013  & 1 776  & 122   & 89  & 5,01\% \\
			Apache Knox          & 25 857  & 3 359  & 23    & 22  & 0,65\% \\
			Phoenix              & 102 397 & 12 909 & 345   & 246 & 1,91\% \\
			\bottomrule
		\end{tabular}
	\end{center}
\end{table}


%%%%%%%%%%%%%%%%%%%%%%%%%%%%%%%%%%%%%%%%%%%%%%%%%%%%%%%%%%%%
\subsection{Om�wienie zebranych danych pomiarowych}

\noindent Zebrane dane zawieraj� 54 kolumny. Pierwsza zawiera sygnatur� metody, kolejne 50 to metryki wymienione w~rozdziale \ref{metryki}. S� to warto�ci liczbowe typu zmiennoprzecinkowego. Kolejne kolumny to: \textit{loc} zawieraj�ca miar� wielko�ci metody (warto�ci liczbowe ca�kowite), \textit{NumOfBugs} zawieraj�ca liczb� b��d�w (warto�ci liczbowe ca�kowite), \textit{Buggy} zawieraj�ca informacje o~tym czy metoda zawiera b��d czy nie (warto�ci \textit{1} lub \textit{0}). �adna z~kolumn nie zawiera nieokre�lonych lub brakuj�cych warto�ci.

Zebrano dane z~29 projekt�w o~otwartych �r�d�ach, w~dw�ch wersjach ka�dego projektu.

\paragraph{Wymagania dla repozytorium danych.} Dane powinny by� zgromadzone w~miejscu, kt�re umo�liwia �atwy dost�p poprzez protok� HTTP, bez uwierzytelniania lub logowania. Jest to istotne z~punktu widzenia wykorzystania narz�dzi do analizy i~przetwarzania tych danych. Pliki danych powinny by� oznaczone sumami kontrolnymi aby by�o mo�liwe sprawdzenie poprawno�ci ich pobierania.

\paragraph{Struktura danych i~zawarto�� zbioru.} Zbi�r metryk zosta� zaprojektowany w~celu umo�liwienia przechowywania danych z~projekt�w r�nego typu. Na najwy�szym poziomie zosta� podzielony na 2 grupy: projekty o~otwartym kodzie i~projekty w�asno�ciowe (odpowiednio katalogi \textit{open source} i~\textit{proprietary}). Katalog na kolejny poziomie zagnie�d�enia okre�la nazw� oprogramowania, a~wewn�trz znajduj� si� arkusze danych w~postaci plik�w CSV. CSV (ang. \textit{Comma-separated values}) jest to plik tekstowy rozdzielany przecinkami. Pliki z~danymi s� nazywane numerem/ nazw� wydania. W~przypadku repozytori�w Git zazwyczaj jest to nazwa tagu. W~pierwszej linii ka�dego pliku znajduj� si� nag��wki kolumn. W~tym samym katalogu co plik CSV powinien znajdowa� si� plik z~sum� kontroln� obliczon� funkcj� skr�tu MD5. Plik z~sum� kontroln� powinien mie� tak� sam� nazw� jak plik z~danymi, z~rozszerzeniem \textit{.md5}.

Na rysunku \ref{struktura-zbioru-danych} przedstawiono og�ln� struktur� zbioru danych wraz z~kilkoma przyk�adowymi projektami.

\begin{figure}[htbp]
	\centering
	\includegraphics[width=0.45\textwidth]{diagrams/repozytorium-astcompare.pdf}
	\caption{Struktura repozytorium metryk AstMetrics}
	\label{struktura-zbioru-danych}
\end{figure}


%%%%%%%%%%%%%%%%%%%%%%%%%%%%%%%%%%%%%%%%%%%%%%%%%%%%%%%%%%%%
\chapter{Modele predykcji i ich ewaluacja}
\label{rozdzial4}
\noindent
\paragraph{LOC.} Przyj�to, �e podstawow� metod� wyszukiwania b��d�w, stanowi�c� punkt odniesienia, by�o przegl�danie kodu �r�d�owego od najmniejszej metody do najwi�kszej. Zatem sortuj�c dane wej�ciowe wed�ug kolumny \textit{LOC} niemalej�co mo�na obliczy� a~nast�pnie wykre�li� skuteczno�� takiego przegl�du. Wykresy na rysunku \ref{wedlug-loc} pokazuj�, �e jest to w~przybli�eniu zale�no�� liniowa, co udowadnia nisk� efektywno�� takich przegl�d�w. Taka metoda przegl�du b�dzie dalej nazywana \textit{LOC}.

\begin{figure}[htbp]
	\centering
	\includegraphics[width=0.49\textwidth]{charts/cost-ant-sorted}
	\includegraphics[width=0.49\textwidth]{charts/cost-commons-lang-sorted}
	\caption{Skuteczno�� przegl�du LOC dla projekt�w Ant i~Commons Lang}
	\label{wedlug-loc}
\end{figure}

\paragraph{Random Forest.} W~licznych badaniach, m.in. \cite{hata2012bug, kamei2010revisiting, lessmann2008benchmarking, liaw2002classification, mende2010effort} wykazano wysok� skuteczno�� metody Random Forest w~predykcji defekt�w. Ponadto udzia� metod z~b��dami jest niewielki, w~zwi�zku z~czym modele predykcji maj� tendencj� do klasyfikowania wszystkich element�w do klasy \textit{0}. Zauwa�ono takie zachowanie podczas stosowania regresji logistycznej w~\cite{hata2012bug}. Problem ten nie dotyka tej metody. Metoda Random Forest polega na tworzeniu du�ej liczby drzew decyzyjnych a~nast�pnie przypisaniu klasy na podstawie dominanty (najcz�stszej warto�ci) spo�r�d wszystkich wynik�w cz�ciowych.


W tabeli \ref{wyniki} zawarto wyniki predykcji z~u�yciem metody LOC oraz Random Forest. Wyniki powy�ej 60\% zaznaczono pogrubion� czcionk�. Uzyskane rezultaty potwierdzaj� wysok� skuteczno�� algorytmu Random Forest. W~kolejnej tabeli \ref{wyniki-rf} umieszczono wyniki pozosta�ych miar skuteczno�ci dla algorytmu Random Forest.

Metoda podstawowa (LOC) cechuje si� efektywno�ci� na poziomie �rednim 27,4\% ($\sigma=10,4$). Wynik uzyskany dla RandomForest wynosi �rednio 54,4\% ($\sigma=9,8$).

\begin{table}[htbp]
	\caption{Efektywno�� predykcji defekt�w pod wzgl�dem wysi�ku. Procent znalezionych b��d�w w~20\% kodu.}
	\label{wyniki}
	\begin{center}
		\tabcolsep=0.11cm
		\begin{tabular}{@{}lrr@{}}
			\toprule
			\bf Nazwa projektu &
			\bf LOC &
			\bf RandomForest
			\\
			\midrule
			ECF                  & 30 & \bf63 \\
			Ant                  & 19 & 49 \\
			JMeter               & 25 & \bf68 \\
			Commons Lang         & 32 & 49 \\
			AspectJ              & 18 & 59 \\
			CXF                  & 43 & 58 \\
			Isis                 & 26 & 32 \\
			Commons Email        & 15 & 54 \\
			Commons JXPath       & 34 & \bf60 \\
			Commons Logging      & 0  & 50 \\
			Commons Net          & 28 & 40 \\
			Accumulo             & 32 & \bf68 \\
			Karaf                & 24 & 43 \\
			ActiveMQ             & 19 & 55 \\
			Hive                 & 40 & 54 \\
			James Mailbox        & 35 & 50 \\
			Jackrabbit FileVault & 38 & \bf64 \\
			OpenJPA              & 34 & 50 \\
			Jackrabbit Oak       & 34 & \bf66 \\
			TomEE                & 19 & 41 \\
			Tika                 & 38 & 46 \\
			Lucene - Core        & 44 & 56 \\
			Solr                 & 43 & 50 \\
			Mahout               & 30 & 52 \\
			Apache Gora          & 9  & \bf64 \\
			Flume                & 18 & 46 \\
			Nutch                & 25 & \bf60 \\
			Apache Knox          & 17 & 52 \\
			Phoenix              & 24 & \bf69 \\
			\bottomrule
		\end{tabular}
	\end{center}
\end{table}



\begin{table}[htbp]
	\caption{Skuteczno�� predykcji defekt�w z~wykorzystaniem algorytmu Random Forest}
	\label{wyniki-rf}
	\begin{center}
		\begin{tabular}{@{}lS[table-format=1.3]S[table-format=1.3]S[table-format=1.3]@{}}
			\toprule
			\bf Nazwa projektu &
			\bf $A$ &
			\bf $\kappa$ &
			\bf $AUC$
			\\
			\midrule
			ECF                  & 0,985 & 0,594 & 0,760 \\
			Ant                  & 0,964 & 0,473 & 0,758 \\
			JMeter               & 0,971 & 0,639 & 0,815 \\
			Commons Lang         & 0,989 & 0,444 & 0,737 \\
			AspectJ              & 0,979 & 0,399 & 0,776 \\
			CXF                  & 0,975 & 0,255 & 0,795 \\
			Isis                 & 0,999 & 0,105 & 0,547 \\
			Commons Email        & 0,957 & 0,000 & 0,697 \\
			Commons JXPath       & 0,988 & 0,622 & 0,790 \\
			Commons Logging      & 0,971 & 0,447 & 0,767 \\
			Commons Net          & 0,941 & 0,320 & 0,757 \\
			Accumulo             & 0,960 & 0,466 & 0,829 \\
			Karaf                & 0,982 & 0,094 & 0,631 \\
			ActiveMQ             & 0,994 & 0,580 & 0,747 \\
			Hive                 & 0,982 & 0,130 & 0,691 \\
			James Mailbox        & 0,991 & 0,329 & 0,783 \\
			Jackrabbit FileVault & 0,990 & 0,057 & 0,637 \\
			OpenJPA              & 0,962 & 0,283 & 0,817 \\
			Jackrabbit Oak       & 0,962 & 0,488 & 0,817 \\
			TomEE                & 0,963 & 0,528 & 0,785 \\
			Tika                 & 0,960 & 0,425 & 0,693 \\
			Lucene - Core        & 0,973 & 0,288 & 0,760 \\
			Solr                 & 0,993 & 0,425 & 0,719 \\
			Mahout               & 0,969 & 0,450 & 0,747 \\
			Apache Gora          & 0,992 & 0,316 & 0,832 \\
			Flume                & 0,977 & 0,323 & 0,704 \\
			Nutch                & 0,967 & 0,582 & 0,788 \\
			Apache Knox          & 0,995 & 0,450 & 0,747 \\
			Phoenix              & 0,978 & 0,461 & 0,875 \\
			\bottomrule
		\end{tabular}
	\end{center}
\end{table}


Na kolejnych rysunkach zamieszczono wykresy krzywej efektywno�ci dla algorytmu Random Forest, dla pierwszych sze�ciu projekt�w. Im bardziej stroma jest krzywa efektywno�ci tym lepsza efektywno�� przeszukiwania kodu �r�d�owego.


\begin{figure}[htbp]
	\centering
	\includegraphics[width=.49\textwidth]{charts/cost-org-eclipse-ecf-rf}
	\includegraphics[width=.49\textwidth]{charts/cost-ant-rf}
	\caption{Krzywa efektywno�ci --- ECF i~Ant}
\end{figure}

\begin{figure}[htbp]
	\centering
	\includegraphics[width=.49\textwidth]{charts/cost-jmeter-rf}
	\includegraphics[width=.49\textwidth]{charts/cost-commons-lang-rf}
	\caption{Krzywa efektywno�ci --- JMeter i~Commons Lang}
\end{figure}

\begin{figure}[htbp]
	\centering
	\includegraphics[width=.49\textwidth]{charts/cost-org-aspectj-rf}
	\includegraphics[width=.49\textwidth]{charts/cost-cxf-rf}
	\caption{Krzywa efektywno�ci --- AspectJ i~CXF}
\end{figure}

Tabele \ref{conf1} - \ref{conf3} zawieraj� macierze pomy�ek dla predykcji defekt�w w~poszczeg�lnych projektach z~wykorzystaniem algorytmu Random Forest.

\begin{table}[htbp]
	\caption{Macierz pomy�ek --- ECF i~Ant}
	\label{conf1}
	\begin{center}
		\begin{tabular}{|c|c|c|c|}
			\cline{3-4}
			\multicolumn{2}{c}{} & \multicolumn{2}{|c|}{przewidywany} \\
			\cline{3-4}
			\multicolumn{2}{c|}{} & 1 & 0 \\
			\cline{1-4}
			\multirow{2}{*}{rzeczywisty} & 1 	& 9060 & 131 \\
			\cline{2-4}
			& 0 								& 10 & 105 \\
			\cline{1-4}
		\end{tabular}
		~
		\begin{tabular}{|c|c|c|c|}
			\cline{3-4}
			\multicolumn{2}{c}{} & \multicolumn{2}{|c|}{przewidywany} \\
			\cline{3-4}
			\multicolumn{2}{c|}{} & 1 & 0 \\
			\cline{1-4}
			\multirow{2}{*}{rzeczywisty} & 1 	& 8642 & 255 \\
			\cline{2-4}
			& 0 								& 71 & 152 \\
			\cline{1-4}
		\end{tabular}
	\end{center}
\end{table}

\begin{table}[htbp]
	\caption{Macierz pomy�ek --- JMeter i~Commons Lang}
	\label{conf2}
	\begin{center}
		\begin{tabular}{|c|c|c|c|}
			\cline{3-4}
			\multicolumn{2}{c}{} & \multicolumn{2}{|c|}{przewidywany} \\
			\cline{3-4}
			\multicolumn{2}{c|}{} & 1 & 0 \\
			\cline{1-4}
			\multirow{2}{*}{rzeczywisty} & 1 	& 208 & 57 \\
			\cline{2-4}
			& 0 								& 157 & 7170 \\
			\cline{1-4}
		\end{tabular}
		~
		\begin{tabular}{|c|c|c|c|}
			\cline{3-4}
			\multicolumn{2}{c}{} & \multicolumn{2}{|c|}{przewidywany} \\
			\cline{3-4}
			\multicolumn{2}{c|}{} & 1 & 0 \\
			\cline{1-4}
			\multirow{2}{*}{rzeczywisty} & 1 	& 11 & 2 \\
			\cline{2-4}
			& 0 								& 23 & 2519 \\
			\cline{1-4}
		\end{tabular}
	\end{center}
\end{table}

\begin{table}[htbp]
	\caption{Macierz pomy�ek --- AspectJ i~CXF}
	\label{conf3}
	\begin{center}
		\begin{tabular}{|c|c|c|c|}
			\cline{3-4}
			\multicolumn{2}{c}{} & \multicolumn{2}{|c|}{przewidywany} \\
			\cline{3-4}
			\multicolumn{2}{c|}{} & 1 & 0 \\
			\cline{1-4}
			\multirow{2}{*}{rzeczywisty} & 1 	& 134 & 72 \\
			\cline{2-4}
			& 0 								& 301 & 17583 \\
			\cline{1-4}
		\end{tabular}
		~
		\begin{tabular}{|c|c|c|c|}
			\cline{3-4}
			\multicolumn{2}{c}{} & \multicolumn{2}{|c|}{przewidywany} \\
			\cline{3-4}
			\multicolumn{2}{c|}{} & 1 & 0 \\
			\cline{1-4}
			\multirow{2}{*}{rzeczywisty} & 1 	& 182 & 409 \\
			\cline{2-4}
			& 0 								& 577 & 38973 \\
			\cline{1-4}
		\end{tabular}
	\end{center}
\end{table}


%%%%%%%%%%%%%%%%%%%%%%%%%%%%%%%%%%%%%%%%%%%%%%%%%%%%%%%%%%%%
\chapter{Podsumowanie}
\label{rozdzial5}
\noindent
\paragraph{Trudno�ci podczas realizacji pracy.} Podczas uruchamiania oblicze� zaplanowanych do wykonania w ramach projektu, okaza�o si�, �e czas przetwarzania na �redniej klasy komputerze osobistym jest zdecydowanie zbyt d�ugi. Po wykonaniu cz�ci oblicze� i zebraniu danych z 6 projekt�w zdecydowano o przeniesieniu wykonania programu do chmury obliczeniowej. Uruchomiono instacj� maszyny wirtualnej w chmurze Windows Azure. Niezb�dne by�o uruchomienie instancji dysponuj�cej odpowiedni� ilo�ci� pami�ci operacyjnej, przydzielono 14 GB pami�ci RAM, z czego 4 GB przeznaczono na przechowywanie plik�w na wirtualnym dysku RAM.

Kolejnym wyzwaniem napotkanym w trakcie tworzenia rozwi�zania by�o wykorzystanie wtyczek z pakietu DePress w �rodowisku KNIME, pomimo przeprowadzenia predykcji i oblicze� statystycznych w �rodowisku R. Dzi�ki wykorzystaniu trybu wsadowego KNIME uzyskano zadowalaj�cy rezultat, daj�cy mo�liwo�� automatycznego przeprowadzenia oblicze�. Wymagane do tego jest jednak zainstalowanie KNIME oraz wtyczek z pakietu DePress.



\paragraph{Zagro�enia dla wiarygodno�ci.} Istniej� zagro�enia dla wiarygodno�ci przeprowadzonego badania. Metoda wyszukiwania b��d�w bazuje na zg�oszeniach w systemie kontroli wersji oraz opisach rewizji w systemie wersjonowania kodu. Od jako�ci tych danych zale�y jako�� uzyskanych informacji o b��dach w kodzie �r�d�owym. Cz�� b��d�w mog�a zosta� �le opisana przez co niemo�liwe by�o dopasowanie rewizji do zg�oszonego b��du. Z drugiej strony cz�� zg�oszonych b��d�w mo�e nie by� b��dami, co tak�e wp�ywa na niedoskona�o�� zebranych danych.

Filtrowanie danych wej�ciowych tak�e mo�e by� nieprecyzyjne. Opracowano rozi�zanie bazuj�ce na wy��czaniu okre�lonego jednego katalogu (razem z podkatalogami), co mo�e by� niewystarczaj�ce ze wzgl�du na uk�ad zawarto�ci niekt�rych repozytori�w.



\paragraph{Podsumowanie.} Stworzone modele predykcji defekt�w na poziomie metod charakteryzuj� si� wysok� skuteczno�ci�. Dzi�ki wykorzystaniu du�ej liczby metryk procesu oraz odpowiedniego algorytmu klasyfikacji, mo�na znacz�co ograniczy� koszty przegl�du kodu, co z~kolei pozwala na ograniczenie koszt�w zapewnienia jako�ci w~procesie tworzenia oprogramowania. Uzyskane wyniki daj� podstawy do pozytywnej oceny stworzonych modeli predykcji. Dzi�ki zastosowaniu opracowanej metody w procesie wytwarzania oprogramowania, czas zaoszcz�dzony na pe�nych przegl�dach kodu, mo�na przeznaczy� na inne zadania realizowane przez zesp� programist�w i tester�w.

W ramach niniejszej pracy dyplomowej stworzono r�wnie� wtyczk� KNIME s�u��c� do wyliczenia metryk procesu AST (AstMetrics). Zbudowano te� repozytorium metryk oraz zebrano metryki z~29 projekt�w. Pozyskane dane mog� pos�u�y� do budowania i~ewaluacji nowych modeli predykcji w~celu uzyskania jeszcze lepszych rezultat�w. Repozytorium metryk opublikowano pod adresem\\https://github.com/mkutyba/AstMetrics-dataset.



\paragraph{Propozycja dalszych bada�.} Zauwa�ono mo�liwo�ci dalszego rozwoju badania w celu osi�gni�cia wi�kszej efektywno�ci. Mo�na dokona� wst�pnej obr�bki zmiennych wej�ciowych, np. stosuj�c metody nadpr�bkowania lub podpr�bkowania albo przefiltrowa� predyktory wy��czaj�c te, kt�re maj� wariancj� blisk� zeru (sta�� warto��), s� skorelowane, lub w inny spos�b.


%%%%%%%%%%%%%%%%%%%%%%%%%%%%%%%%%%%%%%%%%%%%%%%%%%%%%%%%%%%%





 
%\listoffigures
%\listoftables

\bibliographystyle{iisthesis}
\bibliography{bibliografia}

%\appendix
%\chapter{Co� dodatkowego}
%\pagestyle{plain}

\end{document}
